
% Default to the notebook output style

    


% Inherit from the specified cell style.




    
\documentclass[11pt]{article}

    
    
    \usepackage[T1]{fontenc}
    % Nicer default font (+ math font) than Computer Modern for most use cases
    \usepackage{mathpazo}

    % Basic figure setup, for now with no caption control since it's done
    % automatically by Pandoc (which extracts ![](path) syntax from Markdown).
    \usepackage{graphicx}
    % We will generate all images so they have a width \maxwidth. This means
    % that they will get their normal width if they fit onto the page, but
    % are scaled down if they would overflow the margins.
    \makeatletter
    \def\maxwidth{\ifdim\Gin@nat@width>\linewidth\linewidth
    \else\Gin@nat@width\fi}
    \makeatother
    \let\Oldincludegraphics\includegraphics
    % Set max figure width to be 80% of text width, for now hardcoded.
    \renewcommand{\includegraphics}[1]{\Oldincludegraphics[width=.8\maxwidth]{#1}}
    % Ensure that by default, figures have no caption (until we provide a
    % proper Figure object with a Caption API and a way to capture that
    % in the conversion process - todo).
    \usepackage{caption}
    \DeclareCaptionLabelFormat{nolabel}{}
    \captionsetup{labelformat=nolabel}

    \usepackage{adjustbox} % Used to constrain images to a maximum size 
    \usepackage{xcolor} % Allow colors to be defined
    \usepackage{enumerate} % Needed for markdown enumerations to work
    \usepackage{geometry} % Used to adjust the document margins
    \usepackage{amsmath} % Equations
    \usepackage{amssymb} % Equations
    \usepackage{textcomp} % defines textquotesingle
    % Hack from http://tex.stackexchange.com/a/47451/13684:
    \AtBeginDocument{%
        \def\PYZsq{\textquotesingle}% Upright quotes in Pygmentized code
    }
    \usepackage{upquote} % Upright quotes for verbatim code
    \usepackage{eurosym} % defines \euro
    \usepackage[mathletters]{ucs} % Extended unicode (utf-8) support
    \usepackage[utf8x]{inputenc} % Allow utf-8 characters in the tex document
    \usepackage{fancyvrb} % verbatim replacement that allows latex
    \usepackage{grffile} % extends the file name processing of package graphics 
                         % to support a larger range 
    % The hyperref package gives us a pdf with properly built
    % internal navigation ('pdf bookmarks' for the table of contents,
    % internal cross-reference links, web links for URLs, etc.)
    \usepackage{hyperref}
    \usepackage{longtable} % longtable support required by pandoc >1.10
    \usepackage{booktabs}  % table support for pandoc > 1.12.2
    \usepackage[inline]{enumitem} % IRkernel/repr support (it uses the enumerate* environment)
    \usepackage[normalem]{ulem} % ulem is needed to support strikethroughs (\sout)
                                % normalem makes italics be italics, not underlines
    

    
    
    % Colors for the hyperref package
    \definecolor{urlcolor}{rgb}{0,.145,.698}
    \definecolor{linkcolor}{rgb}{.71,0.21,0.01}
    \definecolor{citecolor}{rgb}{.12,.54,.11}

    % ANSI colors
    \definecolor{ansi-black}{HTML}{3E424D}
    \definecolor{ansi-black-intense}{HTML}{282C36}
    \definecolor{ansi-red}{HTML}{E75C58}
    \definecolor{ansi-red-intense}{HTML}{B22B31}
    \definecolor{ansi-green}{HTML}{00A250}
    \definecolor{ansi-green-intense}{HTML}{007427}
    \definecolor{ansi-yellow}{HTML}{DDB62B}
    \definecolor{ansi-yellow-intense}{HTML}{B27D12}
    \definecolor{ansi-blue}{HTML}{208FFB}
    \definecolor{ansi-blue-intense}{HTML}{0065CA}
    \definecolor{ansi-magenta}{HTML}{D160C4}
    \definecolor{ansi-magenta-intense}{HTML}{A03196}
    \definecolor{ansi-cyan}{HTML}{60C6C8}
    \definecolor{ansi-cyan-intense}{HTML}{258F8F}
    \definecolor{ansi-white}{HTML}{C5C1B4}
    \definecolor{ansi-white-intense}{HTML}{A1A6B2}

    % commands and environments needed by pandoc snippets
    % extracted from the output of `pandoc -s`
    \providecommand{\tightlist}{%
      \setlength{\itemsep}{0pt}\setlength{\parskip}{0pt}}
    \DefineVerbatimEnvironment{Highlighting}{Verbatim}{commandchars=\\\{\}}
    % Add ',fontsize=\small' for more characters per line
    \newenvironment{Shaded}{}{}
    \newcommand{\KeywordTok}[1]{\textcolor[rgb]{0.00,0.44,0.13}{\textbf{{#1}}}}
    \newcommand{\DataTypeTok}[1]{\textcolor[rgb]{0.56,0.13,0.00}{{#1}}}
    \newcommand{\DecValTok}[1]{\textcolor[rgb]{0.25,0.63,0.44}{{#1}}}
    \newcommand{\BaseNTok}[1]{\textcolor[rgb]{0.25,0.63,0.44}{{#1}}}
    \newcommand{\FloatTok}[1]{\textcolor[rgb]{0.25,0.63,0.44}{{#1}}}
    \newcommand{\CharTok}[1]{\textcolor[rgb]{0.25,0.44,0.63}{{#1}}}
    \newcommand{\StringTok}[1]{\textcolor[rgb]{0.25,0.44,0.63}{{#1}}}
    \newcommand{\CommentTok}[1]{\textcolor[rgb]{0.38,0.63,0.69}{\textit{{#1}}}}
    \newcommand{\OtherTok}[1]{\textcolor[rgb]{0.00,0.44,0.13}{{#1}}}
    \newcommand{\AlertTok}[1]{\textcolor[rgb]{1.00,0.00,0.00}{\textbf{{#1}}}}
    \newcommand{\FunctionTok}[1]{\textcolor[rgb]{0.02,0.16,0.49}{{#1}}}
    \newcommand{\RegionMarkerTok}[1]{{#1}}
    \newcommand{\ErrorTok}[1]{\textcolor[rgb]{1.00,0.00,0.00}{\textbf{{#1}}}}
    \newcommand{\NormalTok}[1]{{#1}}
    
    % Additional commands for more recent versions of Pandoc
    \newcommand{\ConstantTok}[1]{\textcolor[rgb]{0.53,0.00,0.00}{{#1}}}
    \newcommand{\SpecialCharTok}[1]{\textcolor[rgb]{0.25,0.44,0.63}{{#1}}}
    \newcommand{\VerbatimStringTok}[1]{\textcolor[rgb]{0.25,0.44,0.63}{{#1}}}
    \newcommand{\SpecialStringTok}[1]{\textcolor[rgb]{0.73,0.40,0.53}{{#1}}}
    \newcommand{\ImportTok}[1]{{#1}}
    \newcommand{\DocumentationTok}[1]{\textcolor[rgb]{0.73,0.13,0.13}{\textit{{#1}}}}
    \newcommand{\AnnotationTok}[1]{\textcolor[rgb]{0.38,0.63,0.69}{\textbf{\textit{{#1}}}}}
    \newcommand{\CommentVarTok}[1]{\textcolor[rgb]{0.38,0.63,0.69}{\textbf{\textit{{#1}}}}}
    \newcommand{\VariableTok}[1]{\textcolor[rgb]{0.10,0.09,0.49}{{#1}}}
    \newcommand{\ControlFlowTok}[1]{\textcolor[rgb]{0.00,0.44,0.13}{\textbf{{#1}}}}
    \newcommand{\OperatorTok}[1]{\textcolor[rgb]{0.40,0.40,0.40}{{#1}}}
    \newcommand{\BuiltInTok}[1]{{#1}}
    \newcommand{\ExtensionTok}[1]{{#1}}
    \newcommand{\PreprocessorTok}[1]{\textcolor[rgb]{0.74,0.48,0.00}{{#1}}}
    \newcommand{\AttributeTok}[1]{\textcolor[rgb]{0.49,0.56,0.16}{{#1}}}
    \newcommand{\InformationTok}[1]{\textcolor[rgb]{0.38,0.63,0.69}{\textbf{\textit{{#1}}}}}
    \newcommand{\WarningTok}[1]{\textcolor[rgb]{0.38,0.63,0.69}{\textbf{\textit{{#1}}}}}
    
    
    % Define a nice break command that doesn't care if a line doesn't already
    % exist.
    \def\br{\hspace*{\fill} \\* }
    % Math Jax compatability definitions
    \def\gt{>}
    \def\lt{<}
    % Document parameters
    \title{RNN\_project}
    
    
    

    % Pygments definitions
    
\makeatletter
\def\PY@reset{\let\PY@it=\relax \let\PY@bf=\relax%
    \let\PY@ul=\relax \let\PY@tc=\relax%
    \let\PY@bc=\relax \let\PY@ff=\relax}
\def\PY@tok#1{\csname PY@tok@#1\endcsname}
\def\PY@toks#1+{\ifx\relax#1\empty\else%
    \PY@tok{#1}\expandafter\PY@toks\fi}
\def\PY@do#1{\PY@bc{\PY@tc{\PY@ul{%
    \PY@it{\PY@bf{\PY@ff{#1}}}}}}}
\def\PY#1#2{\PY@reset\PY@toks#1+\relax+\PY@do{#2}}

\expandafter\def\csname PY@tok@s\endcsname{\def\PY@tc##1{\textcolor[rgb]{0.73,0.13,0.13}{##1}}}
\expandafter\def\csname PY@tok@gp\endcsname{\let\PY@bf=\textbf\def\PY@tc##1{\textcolor[rgb]{0.00,0.00,0.50}{##1}}}
\expandafter\def\csname PY@tok@cp\endcsname{\def\PY@tc##1{\textcolor[rgb]{0.74,0.48,0.00}{##1}}}
\expandafter\def\csname PY@tok@nf\endcsname{\def\PY@tc##1{\textcolor[rgb]{0.00,0.00,1.00}{##1}}}
\expandafter\def\csname PY@tok@kr\endcsname{\let\PY@bf=\textbf\def\PY@tc##1{\textcolor[rgb]{0.00,0.50,0.00}{##1}}}
\expandafter\def\csname PY@tok@sh\endcsname{\def\PY@tc##1{\textcolor[rgb]{0.73,0.13,0.13}{##1}}}
\expandafter\def\csname PY@tok@vm\endcsname{\def\PY@tc##1{\textcolor[rgb]{0.10,0.09,0.49}{##1}}}
\expandafter\def\csname PY@tok@si\endcsname{\let\PY@bf=\textbf\def\PY@tc##1{\textcolor[rgb]{0.73,0.40,0.53}{##1}}}
\expandafter\def\csname PY@tok@nn\endcsname{\let\PY@bf=\textbf\def\PY@tc##1{\textcolor[rgb]{0.00,0.00,1.00}{##1}}}
\expandafter\def\csname PY@tok@c\endcsname{\let\PY@it=\textit\def\PY@tc##1{\textcolor[rgb]{0.25,0.50,0.50}{##1}}}
\expandafter\def\csname PY@tok@o\endcsname{\def\PY@tc##1{\textcolor[rgb]{0.40,0.40,0.40}{##1}}}
\expandafter\def\csname PY@tok@sb\endcsname{\def\PY@tc##1{\textcolor[rgb]{0.73,0.13,0.13}{##1}}}
\expandafter\def\csname PY@tok@gi\endcsname{\def\PY@tc##1{\textcolor[rgb]{0.00,0.63,0.00}{##1}}}
\expandafter\def\csname PY@tok@nb\endcsname{\def\PY@tc##1{\textcolor[rgb]{0.00,0.50,0.00}{##1}}}
\expandafter\def\csname PY@tok@m\endcsname{\def\PY@tc##1{\textcolor[rgb]{0.40,0.40,0.40}{##1}}}
\expandafter\def\csname PY@tok@mh\endcsname{\def\PY@tc##1{\textcolor[rgb]{0.40,0.40,0.40}{##1}}}
\expandafter\def\csname PY@tok@gh\endcsname{\let\PY@bf=\textbf\def\PY@tc##1{\textcolor[rgb]{0.00,0.00,0.50}{##1}}}
\expandafter\def\csname PY@tok@mf\endcsname{\def\PY@tc##1{\textcolor[rgb]{0.40,0.40,0.40}{##1}}}
\expandafter\def\csname PY@tok@err\endcsname{\def\PY@bc##1{\setlength{\fboxsep}{0pt}\fcolorbox[rgb]{1.00,0.00,0.00}{1,1,1}{\strut ##1}}}
\expandafter\def\csname PY@tok@c1\endcsname{\let\PY@it=\textit\def\PY@tc##1{\textcolor[rgb]{0.25,0.50,0.50}{##1}}}
\expandafter\def\csname PY@tok@gt\endcsname{\def\PY@tc##1{\textcolor[rgb]{0.00,0.27,0.87}{##1}}}
\expandafter\def\csname PY@tok@na\endcsname{\def\PY@tc##1{\textcolor[rgb]{0.49,0.56,0.16}{##1}}}
\expandafter\def\csname PY@tok@ni\endcsname{\let\PY@bf=\textbf\def\PY@tc##1{\textcolor[rgb]{0.60,0.60,0.60}{##1}}}
\expandafter\def\csname PY@tok@kt\endcsname{\def\PY@tc##1{\textcolor[rgb]{0.69,0.00,0.25}{##1}}}
\expandafter\def\csname PY@tok@se\endcsname{\let\PY@bf=\textbf\def\PY@tc##1{\textcolor[rgb]{0.73,0.40,0.13}{##1}}}
\expandafter\def\csname PY@tok@go\endcsname{\def\PY@tc##1{\textcolor[rgb]{0.53,0.53,0.53}{##1}}}
\expandafter\def\csname PY@tok@nt\endcsname{\let\PY@bf=\textbf\def\PY@tc##1{\textcolor[rgb]{0.00,0.50,0.00}{##1}}}
\expandafter\def\csname PY@tok@il\endcsname{\def\PY@tc##1{\textcolor[rgb]{0.40,0.40,0.40}{##1}}}
\expandafter\def\csname PY@tok@fm\endcsname{\def\PY@tc##1{\textcolor[rgb]{0.00,0.00,1.00}{##1}}}
\expandafter\def\csname PY@tok@sa\endcsname{\def\PY@tc##1{\textcolor[rgb]{0.73,0.13,0.13}{##1}}}
\expandafter\def\csname PY@tok@gu\endcsname{\let\PY@bf=\textbf\def\PY@tc##1{\textcolor[rgb]{0.50,0.00,0.50}{##1}}}
\expandafter\def\csname PY@tok@k\endcsname{\let\PY@bf=\textbf\def\PY@tc##1{\textcolor[rgb]{0.00,0.50,0.00}{##1}}}
\expandafter\def\csname PY@tok@bp\endcsname{\def\PY@tc##1{\textcolor[rgb]{0.00,0.50,0.00}{##1}}}
\expandafter\def\csname PY@tok@vc\endcsname{\def\PY@tc##1{\textcolor[rgb]{0.10,0.09,0.49}{##1}}}
\expandafter\def\csname PY@tok@w\endcsname{\def\PY@tc##1{\textcolor[rgb]{0.73,0.73,0.73}{##1}}}
\expandafter\def\csname PY@tok@vi\endcsname{\def\PY@tc##1{\textcolor[rgb]{0.10,0.09,0.49}{##1}}}
\expandafter\def\csname PY@tok@sx\endcsname{\def\PY@tc##1{\textcolor[rgb]{0.00,0.50,0.00}{##1}}}
\expandafter\def\csname PY@tok@kn\endcsname{\let\PY@bf=\textbf\def\PY@tc##1{\textcolor[rgb]{0.00,0.50,0.00}{##1}}}
\expandafter\def\csname PY@tok@cpf\endcsname{\let\PY@it=\textit\def\PY@tc##1{\textcolor[rgb]{0.25,0.50,0.50}{##1}}}
\expandafter\def\csname PY@tok@kp\endcsname{\def\PY@tc##1{\textcolor[rgb]{0.00,0.50,0.00}{##1}}}
\expandafter\def\csname PY@tok@ss\endcsname{\def\PY@tc##1{\textcolor[rgb]{0.10,0.09,0.49}{##1}}}
\expandafter\def\csname PY@tok@gs\endcsname{\let\PY@bf=\textbf}
\expandafter\def\csname PY@tok@kd\endcsname{\let\PY@bf=\textbf\def\PY@tc##1{\textcolor[rgb]{0.00,0.50,0.00}{##1}}}
\expandafter\def\csname PY@tok@nd\endcsname{\def\PY@tc##1{\textcolor[rgb]{0.67,0.13,1.00}{##1}}}
\expandafter\def\csname PY@tok@ch\endcsname{\let\PY@it=\textit\def\PY@tc##1{\textcolor[rgb]{0.25,0.50,0.50}{##1}}}
\expandafter\def\csname PY@tok@cs\endcsname{\let\PY@it=\textit\def\PY@tc##1{\textcolor[rgb]{0.25,0.50,0.50}{##1}}}
\expandafter\def\csname PY@tok@dl\endcsname{\def\PY@tc##1{\textcolor[rgb]{0.73,0.13,0.13}{##1}}}
\expandafter\def\csname PY@tok@mi\endcsname{\def\PY@tc##1{\textcolor[rgb]{0.40,0.40,0.40}{##1}}}
\expandafter\def\csname PY@tok@ow\endcsname{\let\PY@bf=\textbf\def\PY@tc##1{\textcolor[rgb]{0.67,0.13,1.00}{##1}}}
\expandafter\def\csname PY@tok@ne\endcsname{\let\PY@bf=\textbf\def\PY@tc##1{\textcolor[rgb]{0.82,0.25,0.23}{##1}}}
\expandafter\def\csname PY@tok@gr\endcsname{\def\PY@tc##1{\textcolor[rgb]{1.00,0.00,0.00}{##1}}}
\expandafter\def\csname PY@tok@s1\endcsname{\def\PY@tc##1{\textcolor[rgb]{0.73,0.13,0.13}{##1}}}
\expandafter\def\csname PY@tok@nl\endcsname{\def\PY@tc##1{\textcolor[rgb]{0.63,0.63,0.00}{##1}}}
\expandafter\def\csname PY@tok@s2\endcsname{\def\PY@tc##1{\textcolor[rgb]{0.73,0.13,0.13}{##1}}}
\expandafter\def\csname PY@tok@nv\endcsname{\def\PY@tc##1{\textcolor[rgb]{0.10,0.09,0.49}{##1}}}
\expandafter\def\csname PY@tok@sc\endcsname{\def\PY@tc##1{\textcolor[rgb]{0.73,0.13,0.13}{##1}}}
\expandafter\def\csname PY@tok@sd\endcsname{\let\PY@it=\textit\def\PY@tc##1{\textcolor[rgb]{0.73,0.13,0.13}{##1}}}
\expandafter\def\csname PY@tok@ge\endcsname{\let\PY@it=\textit}
\expandafter\def\csname PY@tok@vg\endcsname{\def\PY@tc##1{\textcolor[rgb]{0.10,0.09,0.49}{##1}}}
\expandafter\def\csname PY@tok@cm\endcsname{\let\PY@it=\textit\def\PY@tc##1{\textcolor[rgb]{0.25,0.50,0.50}{##1}}}
\expandafter\def\csname PY@tok@mb\endcsname{\def\PY@tc##1{\textcolor[rgb]{0.40,0.40,0.40}{##1}}}
\expandafter\def\csname PY@tok@gd\endcsname{\def\PY@tc##1{\textcolor[rgb]{0.63,0.00,0.00}{##1}}}
\expandafter\def\csname PY@tok@kc\endcsname{\let\PY@bf=\textbf\def\PY@tc##1{\textcolor[rgb]{0.00,0.50,0.00}{##1}}}
\expandafter\def\csname PY@tok@nc\endcsname{\let\PY@bf=\textbf\def\PY@tc##1{\textcolor[rgb]{0.00,0.00,1.00}{##1}}}
\expandafter\def\csname PY@tok@sr\endcsname{\def\PY@tc##1{\textcolor[rgb]{0.73,0.40,0.53}{##1}}}
\expandafter\def\csname PY@tok@no\endcsname{\def\PY@tc##1{\textcolor[rgb]{0.53,0.00,0.00}{##1}}}
\expandafter\def\csname PY@tok@mo\endcsname{\def\PY@tc##1{\textcolor[rgb]{0.40,0.40,0.40}{##1}}}

\def\PYZbs{\char`\\}
\def\PYZus{\char`\_}
\def\PYZob{\char`\{}
\def\PYZcb{\char`\}}
\def\PYZca{\char`\^}
\def\PYZam{\char`\&}
\def\PYZlt{\char`\<}
\def\PYZgt{\char`\>}
\def\PYZsh{\char`\#}
\def\PYZpc{\char`\%}
\def\PYZdl{\char`\$}
\def\PYZhy{\char`\-}
\def\PYZsq{\char`\'}
\def\PYZdq{\char`\"}
\def\PYZti{\char`\~}
% for compatibility with earlier versions
\def\PYZat{@}
\def\PYZlb{[}
\def\PYZrb{]}
\makeatother


    % Exact colors from NB
    \definecolor{incolor}{rgb}{0.0, 0.0, 0.5}
    \definecolor{outcolor}{rgb}{0.545, 0.0, 0.0}



    
    % Prevent overflowing lines due to hard-to-break entities
    \sloppy 
    % Setup hyperref package
    \hypersetup{
      breaklinks=true,  % so long urls are correctly broken across lines
      colorlinks=true,
      urlcolor=urlcolor,
      linkcolor=linkcolor,
      citecolor=citecolor,
      }
    % Slightly bigger margins than the latex defaults
    
    \geometry{verbose,tmargin=1in,bmargin=1in,lmargin=1in,rmargin=1in}
    
    

    \begin{document}
    
    
    \maketitle
    
    

    
    \section{Artificial Intelligence
Nanodegree}\label{artificial-intelligence-nanodegree}

\subsection{Recurrent Neural Network
Projects}\label{recurrent-neural-network-projects}

Welcome to the Recurrent Neural Network Project in the Artificial
Intelligence Nanodegree! In this notebook, some template code has
already been provided for you, and you will need to implement additional
functionality to successfully complete this project. You will not need
to modify the included code beyond what is requested. Sections that
begin with \textbf{'Implementation'} in the header indicate that the
following block of code will require additional functionality which you
must provide. Instructions will be provided for each section and the
specifics of the implementation are marked in the code block with a
'TODO' statement. Please be sure to read the instructions carefully!

\begin{quote}
\textbf{Note:} Code and Markdown cells can be executed using the
\textbf{Shift + Enter} keyboard shortcut. In addition, Markdown cells
can be edited by typically double-clicking the cell to enter edit mode.
\end{quote}

    \subsubsection{Implementation TODOs in this
notebook}\label{implementation-todos-in-this-notebook}

This notebook contains two problems, cut into a variety of TODOs. Make
sure to complete each section containing a TODO marker throughout the
notebook. For convenience we provide links to each of these sections
below.

Section \ref{todo_1}

Section \ref{todo_2}

Section \ref{todo_3}

Section \ref{todo_4}

Section \ref{todo_5}

Section \ref{todo_6}

    \section{Problem 1: Perform time series
prediction}\label{problem-1-perform-time-series-prediction}

In this project you will perform time series prediction using a
Recurrent Neural Network regressor. In particular you will re-create the
figure shown in the notes - where the stock price of Apple was
forecasted (or predicted) 7 days in advance. In completing this exercise
you will learn how to construct RNNs using Keras, which will also aid in
completing the second project in this notebook.

The particular network architecture we will employ for our RNN is known
as \href{https://en.wikipedia.org/wiki/Long_short-term_memory}{Long Term
Short Memory (LSTM)}, which helps significantly avoid technical problems
with optimization of RNNs.

    \subsection{1.1 Getting started}\label{getting-started}

First we must load in our time series - a history of around 140 days of
Apple's stock price. Then we need to perform a number of pre-processing
steps to prepare it for use with an RNN model. First off, it is good
practice to normalize time series - by normalizing its range. This helps
us avoid serious numerical issues associated how common activation
functions (like tanh) transform very large (positive or negative)
numbers, as well as helping us to avoid related issues when computing
derivatives.

Here we normalize the series to lie in the range {[}0,1{]} {[}using this
scikit
function{]}(http://scikit-learn.org/stable/modules/generated/sklearn.preprocessing.MinMaxScaler.html),
but it is also commonplace to normalize by a series standard deviation.

    \begin{Verbatim}[commandchars=\\\{\}]
{\color{incolor}In [{\color{incolor}1}]:} \PY{c+c1}{\PYZsh{}\PYZsh{}\PYZsh{} Load in necessary libraries for data input and normalization}
        \PY{o}{\PYZpc{}}\PY{k}{matplotlib} inline
        \PY{k+kn}{import} \PY{n+nn}{numpy} \PY{k}{as} \PY{n+nn}{np}
        \PY{k+kn}{import} \PY{n+nn}{matplotlib}\PY{n+nn}{.}\PY{n+nn}{pyplot} \PY{k}{as} \PY{n+nn}{plt}
        
        \PY{o}{\PYZpc{}}\PY{k}{load\PYZus{}ext} autoreload
        \PY{o}{\PYZpc{}}\PY{k}{autoreload} 2
        
        \PY{k+kn}{from} \PY{n+nn}{my\PYZus{}answers} \PY{k}{import} \PY{o}{*}
        
        \PY{o}{\PYZpc{}}\PY{k}{load\PYZus{}ext} autoreload
        \PY{o}{\PYZpc{}}\PY{k}{autoreload} 2
        
        \PY{k+kn}{from} \PY{n+nn}{my\PYZus{}answers} \PY{k}{import} \PY{o}{*}
        
        \PY{c+c1}{\PYZsh{}\PYZsh{}\PYZsh{} load in and normalize the dataset}
        \PY{n}{dataset} \PY{o}{=} \PY{n}{np}\PY{o}{.}\PY{n}{loadtxt}\PY{p}{(}\PY{l+s+s1}{\PYZsq{}}\PY{l+s+s1}{datasets/normalized\PYZus{}apple\PYZus{}prices.csv}\PY{l+s+s1}{\PYZsq{}}\PY{p}{)}
\end{Verbatim}


    \begin{Verbatim}[commandchars=\\\{\}]
C:\textbackslash{}Users\textbackslash{}RickS\textbackslash{}Documents\textbackslash{}utilities\textbackslash{}Miniconda3\textbackslash{}envs\textbackslash{}nb35\textbackslash{}lib\textbackslash{}site-packages\textbackslash{}h5py\textbackslash{}\_\_init\_\_.py:36: FutureWarning: Conversion of the second argument of issubdtype from `float` to `np.floating` is deprecated. In future, it will be treated as `np.float64 == np.dtype(float).type`.
  from .\_conv import register\_converters as \_register\_converters
Using TensorFlow backend.

    \end{Verbatim}

    \begin{Verbatim}[commandchars=\\\{\}]
The autoreload extension is already loaded. To reload it, use:
  \%reload\_ext autoreload

    \end{Verbatim}

    Lets take a quick look at the (normalized) time series we'll be
performing predictions on.

    \begin{Verbatim}[commandchars=\\\{\}]
{\color{incolor}In [{\color{incolor}2}]:} \PY{c+c1}{\PYZsh{} lets take a look at our time series}
        \PY{n}{plt}\PY{o}{.}\PY{n}{plot}\PY{p}{(}\PY{n}{dataset}\PY{p}{)}
        \PY{n}{plt}\PY{o}{.}\PY{n}{xlabel}\PY{p}{(}\PY{l+s+s1}{\PYZsq{}}\PY{l+s+s1}{time period}\PY{l+s+s1}{\PYZsq{}}\PY{p}{)}
        \PY{n}{plt}\PY{o}{.}\PY{n}{ylabel}\PY{p}{(}\PY{l+s+s1}{\PYZsq{}}\PY{l+s+s1}{normalized series value}\PY{l+s+s1}{\PYZsq{}}\PY{p}{)}
\end{Verbatim}


\begin{Verbatim}[commandchars=\\\{\}]
{\color{outcolor}Out[{\color{outcolor}2}]:} <matplotlib.text.Text at 0x22f4c802390>
\end{Verbatim}
            
    \begin{center}
    \adjustimage{max size={0.9\linewidth}{0.9\paperheight}}{output_6_1.png}
    \end{center}
    { \hspace*{\fill} \\}
    
    \subsection{1.2 Cutting our time series into
sequences}\label{cutting-our-time-series-into-sequences}

Remember, our time series is a sequence of numbers that we can represent
in general mathematically as

\[s_{0},s_{1},s_{2},...,s_{P}\]

where \(s_{p}\) is the numerical value of the time series at time period
\(p\) and where \(P\) is the total length of the series. In order to
apply our RNN we treat the time series prediction problem as a
regression problem, and so need to use a sliding window to construct a
set of associated input/output pairs to regress on. This process is
animated in the gif below.

For example - using a window of size T = 5 (as illustrated in the gif
above) we produce a set of input/output pairs like the one shown in the
table below

\[\begin{array}{c|c}
\text{Input} & \text{Output}\\
\hline \color{CornflowerBlue} {\langle s_{1},s_{2},s_{3},s_{4},s_{5}\rangle} & \color{Goldenrod}{ s_{6}} \\
\ \color{CornflowerBlue} {\langle s_{2},s_{3},s_{4},s_{5},s_{6} \rangle } & \color{Goldenrod} {s_{7} } \\
\color{CornflowerBlue}  {\vdots} & \color{Goldenrod} {\vdots}\\
\color{CornflowerBlue} { \langle s_{P-5},s_{P-4},s_{P-3},s_{P-2},s_{P-1} \rangle } & \color{Goldenrod} {s_{P}}
\end{array}\]

Notice here that each input is a sequence (or vector) of length 5 (and
in general has length equal to the window size T) while each
corresponding output is a scalar value. Notice also how given a time
series of length P and window size T = 5 as shown above, we created P -
5 input/output pairs. More generally, for a window size T we create P -
T such pairs.

    Now its time for you to window the input time series as described above!

\textbf{TODO:} Implement the function called
\textbf{window\_transform\_series} in my\_answers.py so that it runs a
sliding window along the input series and creates associated
input/output pairs. Note that this function should input a) the series
and b) the window length, and return the input/output subsequences. Make
sure to format returned input/output as generally shown in table above
(where window\_size = 5), and make sure your returned input is a numpy
array.

\begin{center}\rule{0.5\linewidth}{\linethickness}\end{center}

    You can test your function on the list of odd numbers given below

    \begin{Verbatim}[commandchars=\\\{\}]
{\color{incolor}In [{\color{incolor}3}]:} \PY{n}{odd\PYZus{}nums} \PY{o}{=} \PY{n}{np}\PY{o}{.}\PY{n}{array}\PY{p}{(}\PY{p}{[}\PY{l+m+mi}{1}\PY{p}{,}\PY{l+m+mi}{3}\PY{p}{,}\PY{l+m+mi}{5}\PY{p}{,}\PY{l+m+mi}{7}\PY{p}{,}\PY{l+m+mi}{9}\PY{p}{,}\PY{l+m+mi}{11}\PY{p}{,}\PY{l+m+mi}{13}\PY{p}{]}\PY{p}{)}
\end{Verbatim}


    Here is a hard-coded solution for odd\_nums. You can compare its results
with what you get from your \textbf{window\_transform\_series}
implementation.

    \begin{Verbatim}[commandchars=\\\{\}]
{\color{incolor}In [{\color{incolor}4}]:} \PY{c+c1}{\PYZsh{} run a window of size 2 over the odd number sequence and display the results}
        \PY{n}{window\PYZus{}size} \PY{o}{=} \PY{l+m+mi}{2}
        
        \PY{n}{X} \PY{o}{=} \PY{p}{[}\PY{p}{]}
        \PY{n}{X}\PY{o}{.}\PY{n}{append}\PY{p}{(}\PY{n}{odd\PYZus{}nums}\PY{p}{[}\PY{l+m+mi}{0}\PY{p}{:}\PY{l+m+mi}{2}\PY{p}{]}\PY{p}{)}
        \PY{n}{X}\PY{o}{.}\PY{n}{append}\PY{p}{(}\PY{n}{odd\PYZus{}nums}\PY{p}{[}\PY{l+m+mi}{1}\PY{p}{:}\PY{l+m+mi}{3}\PY{p}{]}\PY{p}{)}
        \PY{n}{X}\PY{o}{.}\PY{n}{append}\PY{p}{(}\PY{n}{odd\PYZus{}nums}\PY{p}{[}\PY{l+m+mi}{2}\PY{p}{:}\PY{l+m+mi}{4}\PY{p}{]}\PY{p}{)}
        \PY{n}{X}\PY{o}{.}\PY{n}{append}\PY{p}{(}\PY{n}{odd\PYZus{}nums}\PY{p}{[}\PY{l+m+mi}{3}\PY{p}{:}\PY{l+m+mi}{5}\PY{p}{]}\PY{p}{)}
        \PY{n}{X}\PY{o}{.}\PY{n}{append}\PY{p}{(}\PY{n}{odd\PYZus{}nums}\PY{p}{[}\PY{l+m+mi}{4}\PY{p}{:}\PY{l+m+mi}{6}\PY{p}{]}\PY{p}{)}
        
        \PY{n}{y} \PY{o}{=} \PY{n}{odd\PYZus{}nums}\PY{p}{[}\PY{l+m+mi}{2}\PY{p}{:}\PY{p}{]}
        
        \PY{n}{X} \PY{o}{=} \PY{n}{np}\PY{o}{.}\PY{n}{asarray}\PY{p}{(}\PY{n}{X}\PY{p}{)}
        \PY{n}{y} \PY{o}{=} \PY{n}{np}\PY{o}{.}\PY{n}{asarray}\PY{p}{(}\PY{n}{y}\PY{p}{)}
        \PY{n}{y} \PY{o}{=} \PY{n}{np}\PY{o}{.}\PY{n}{reshape}\PY{p}{(}\PY{n}{y}\PY{p}{,} \PY{p}{(}\PY{n+nb}{len}\PY{p}{(}\PY{n}{y}\PY{p}{)}\PY{p}{,}\PY{l+m+mi}{1}\PY{p}{)}\PY{p}{)} \PY{c+c1}{\PYZsh{}optional}
        
        \PY{k}{assert}\PY{p}{(}\PY{n+nb}{type}\PY{p}{(}\PY{n}{X}\PY{p}{)}\PY{o}{.}\PY{n+nv+vm}{\PYZus{}\PYZus{}name\PYZus{}\PYZus{}} \PY{o}{==} \PY{l+s+s1}{\PYZsq{}}\PY{l+s+s1}{ndarray}\PY{l+s+s1}{\PYZsq{}}\PY{p}{)}
        \PY{k}{assert}\PY{p}{(}\PY{n+nb}{type}\PY{p}{(}\PY{n}{y}\PY{p}{)}\PY{o}{.}\PY{n+nv+vm}{\PYZus{}\PYZus{}name\PYZus{}\PYZus{}} \PY{o}{==} \PY{l+s+s1}{\PYZsq{}}\PY{l+s+s1}{ndarray}\PY{l+s+s1}{\PYZsq{}}\PY{p}{)}
        \PY{k}{assert}\PY{p}{(}\PY{n}{X}\PY{o}{.}\PY{n}{shape} \PY{o}{==} \PY{p}{(}\PY{l+m+mi}{5}\PY{p}{,}\PY{l+m+mi}{2}\PY{p}{)}\PY{p}{)}
        \PY{k}{assert}\PY{p}{(}\PY{n}{y}\PY{o}{.}\PY{n}{shape} \PY{o+ow}{in} \PY{p}{[}\PY{p}{(}\PY{l+m+mi}{5}\PY{p}{,}\PY{l+m+mi}{1}\PY{p}{)}\PY{p}{,} \PY{p}{(}\PY{l+m+mi}{5}\PY{p}{,}\PY{p}{)}\PY{p}{]}\PY{p}{)}
        
        \PY{c+c1}{\PYZsh{} print out input/output pairs \PYZhy{}\PYZhy{}\PYZgt{} here input = X, corresponding output = y}
        \PY{n+nb}{print} \PY{p}{(}\PY{l+s+s1}{\PYZsq{}}\PY{l+s+s1}{\PYZhy{}\PYZhy{}\PYZhy{} the input X will look like \PYZhy{}\PYZhy{}\PYZhy{}\PYZhy{}}\PY{l+s+s1}{\PYZsq{}}\PY{p}{)}
        \PY{n+nb}{print} \PY{p}{(}\PY{n}{X}\PY{p}{)}
        
        \PY{n+nb}{print} \PY{p}{(}\PY{l+s+s1}{\PYZsq{}}\PY{l+s+s1}{\PYZhy{}\PYZhy{}\PYZhy{} the associated output y will look like \PYZhy{}\PYZhy{}\PYZhy{}\PYZhy{}}\PY{l+s+s1}{\PYZsq{}}\PY{p}{)}
        \PY{n+nb}{print} \PY{p}{(}\PY{n}{y}\PY{p}{)}
\end{Verbatim}


    \begin{Verbatim}[commandchars=\\\{\}]
--- the input X will look like ----
[[ 1  3]
 [ 3  5]
 [ 5  7]
 [ 7  9]
 [ 9 11]]
--- the associated output y will look like ----
[[ 5]
 [ 7]
 [ 9]
 [11]
 [13]]

    \end{Verbatim}

    Again - you can check that your completed
\textbf{window\_transform\_series} function works correctly by trying it
on the odd\_nums sequence - you should get the above output.

    \begin{Verbatim}[commandchars=\\\{\}]
{\color{incolor}In [{\color{incolor}5}]:} \PY{c+c1}{\PYZsh{}\PYZsh{}\PYZsh{} TODO: implement the function window\PYZus{}transform\PYZus{}series in the file my\PYZus{}answers.py}
        \PY{k+kn}{from} \PY{n+nn}{my\PYZus{}answers} \PY{k}{import} \PY{n}{window\PYZus{}transform\PYZus{}series}
\end{Verbatim}


    With this function in place apply it to the series in the Python cell
below. We use a window\_size = 7 for these experiments.

    \begin{Verbatim}[commandchars=\\\{\}]
{\color{incolor}In [{\color{incolor}6}]:} \PY{c+c1}{\PYZsh{} window the data using your windowing function}
        \PY{n}{window\PYZus{}size} \PY{o}{=} \PY{l+m+mi}{7}
        \PY{n}{X}\PY{p}{,}\PY{n}{y} \PY{o}{=} \PY{n}{window\PYZus{}transform\PYZus{}series}\PY{p}{(}\PY{n}{series} \PY{o}{=} \PY{n}{dataset}\PY{p}{,} \PY{n}{window\PYZus{}size} \PY{o}{=} \PY{n}{window\PYZus{}size}\PY{p}{)}
        \PY{k}{assert}\PY{p}{(}\PY{n}{X}\PY{o}{.}\PY{n}{shape} \PY{o}{==} \PY{p}{(}\PY{n+nb}{len}\PY{p}{(}\PY{n}{dataset}\PY{p}{)}\PY{o}{\PYZhy{}}\PY{n}{window\PYZus{}size}\PY{p}{,} \PY{n}{window\PYZus{}size}\PY{p}{)}\PY{p}{)}
        \PY{k}{assert}\PY{p}{(}\PY{n}{y}\PY{o}{.}\PY{n}{shape} \PY{o+ow}{in} \PY{p}{[}\PY{p}{(}\PY{n+nb}{len}\PY{p}{(}\PY{n}{dataset}\PY{p}{)}\PY{o}{\PYZhy{}}\PY{n}{window\PYZus{}size}\PY{p}{,} \PY{l+m+mi}{1}\PY{p}{)}\PY{p}{,} \PY{p}{(}\PY{n+nb}{len}\PY{p}{(}\PY{n}{dataset}\PY{p}{)}\PY{o}{\PYZhy{}}\PY{n}{window\PYZus{}size}\PY{p}{,}\PY{p}{)}\PY{p}{]}\PY{p}{)}
\end{Verbatim}


    \subsection{1.3 Splitting into training and testing
sets}\label{splitting-into-training-and-testing-sets}

In order to perform proper testing on our dataset we will lop off the
last 1/3 of it for validation (or testing). This is that once we train
our model we have something to test it on (like any regression
problem!). This splitting into training/testing sets is done in the cell
below.

Note how here we are \textbf{not} splitting the dataset \emph{randomly}
as one typically would do when validating a regression model. This is
because our input/output pairs \emph{are related temporally}. We don't
want to validate our model by training on a random subset of the series
and then testing on another random subset, as this simulates the
scenario that we receive new points \emph{within the timeframe of our
training set}.

We want to train on one solid chunk of the series (in our case, the
first full 2/3 of it), and validate on a later chunk (the last 1/3) as
this simulates how we would predict \emph{future} values of a time
series.

    \begin{Verbatim}[commandchars=\\\{\}]
{\color{incolor}In [{\color{incolor}7}]:} \PY{c+c1}{\PYZsh{} split our dataset into training / testing sets}
        \PY{n}{train\PYZus{}test\PYZus{}split} \PY{o}{=} \PY{n+nb}{int}\PY{p}{(}\PY{n}{np}\PY{o}{.}\PY{n}{ceil}\PY{p}{(}\PY{l+m+mi}{2}\PY{o}{*}\PY{n+nb}{len}\PY{p}{(}\PY{n}{y}\PY{p}{)}\PY{o}{/}\PY{n+nb}{float}\PY{p}{(}\PY{l+m+mi}{3}\PY{p}{)}\PY{p}{)}\PY{p}{)}   \PY{c+c1}{\PYZsh{} set the split point}
        
        \PY{c+c1}{\PYZsh{} partition the training set}
        \PY{n}{X\PYZus{}train} \PY{o}{=} \PY{n}{X}\PY{p}{[}\PY{p}{:}\PY{n}{train\PYZus{}test\PYZus{}split}\PY{p}{,}\PY{p}{:}\PY{p}{]}
        \PY{n}{y\PYZus{}train} \PY{o}{=} \PY{n}{y}\PY{p}{[}\PY{p}{:}\PY{n}{train\PYZus{}test\PYZus{}split}\PY{p}{]}
        
        \PY{c+c1}{\PYZsh{} keep the last chunk for testing}
        \PY{n}{X\PYZus{}test} \PY{o}{=} \PY{n}{X}\PY{p}{[}\PY{n}{train\PYZus{}test\PYZus{}split}\PY{p}{:}\PY{p}{,}\PY{p}{:}\PY{p}{]}
        \PY{n}{y\PYZus{}test} \PY{o}{=} \PY{n}{y}\PY{p}{[}\PY{n}{train\PYZus{}test\PYZus{}split}\PY{p}{:}\PY{p}{]}
        
        \PY{c+c1}{\PYZsh{} NOTE: to use keras\PYZsq{}s RNN LSTM module our input must be reshaped to [samples, window size, stepsize] }
        \PY{n}{X\PYZus{}train} \PY{o}{=} \PY{n}{np}\PY{o}{.}\PY{n}{asarray}\PY{p}{(}\PY{n}{np}\PY{o}{.}\PY{n}{reshape}\PY{p}{(}\PY{n}{X\PYZus{}train}\PY{p}{,} \PY{p}{(}\PY{n}{X\PYZus{}train}\PY{o}{.}\PY{n}{shape}\PY{p}{[}\PY{l+m+mi}{0}\PY{p}{]}\PY{p}{,} \PY{n}{window\PYZus{}size}\PY{p}{,} \PY{l+m+mi}{1}\PY{p}{)}\PY{p}{)}\PY{p}{)}
        \PY{n}{X\PYZus{}test} \PY{o}{=} \PY{n}{np}\PY{o}{.}\PY{n}{asarray}\PY{p}{(}\PY{n}{np}\PY{o}{.}\PY{n}{reshape}\PY{p}{(}\PY{n}{X\PYZus{}test}\PY{p}{,} \PY{p}{(}\PY{n}{X\PYZus{}test}\PY{o}{.}\PY{n}{shape}\PY{p}{[}\PY{l+m+mi}{0}\PY{p}{]}\PY{p}{,} \PY{n}{window\PYZus{}size}\PY{p}{,} \PY{l+m+mi}{1}\PY{p}{)}\PY{p}{)}\PY{p}{)}
\end{Verbatim}


    \subsection{1.4 Build and run an RNN regression
model}\label{build-and-run-an-rnn-regression-model}

Having created input/output pairs out of our time series and cut this
into training/testing sets, we can now begin setting up our RNN. We use
Keras to quickly build a two hidden layer RNN of the following
specifications

\begin{itemize}
\tightlist
\item
  layer 1 uses an LSTM module with 5 hidden units (note here the
  input\_shape = (window\_size,1))
\item
  layer 2 uses a fully connected module with one unit
\item
  the 'mean\_squared\_error' loss should be used (remember: we are
  performing regression here)
\end{itemize}

This can be constructed using just a few lines - see e.g., the
\href{https://keras.io/getting-started/sequential-model-guide/}{general
Keras documentation} and the
\href{https://keras.io/layers/recurrent/}{LSTM documentation in
particular} for examples of how to quickly use Keras to build neural
network models. Make sure you are initializing your optimizer given the
\href{https://keras.io/optimizers/}{keras-recommended approach for RNNs}

(given in the cell below). (remember to copy your completed function
into the script \emph{my\_answers.py} function titled
\emph{build\_part1\_RNN} before submitting your project)

    \begin{Verbatim}[commandchars=\\\{\}]
{\color{incolor}In [{\color{incolor}8}]:} \PY{c+c1}{\PYZsh{}\PYZsh{}\PYZsh{} TODO: create required RNN model}
        \PY{c+c1}{\PYZsh{} import keras network libraries}
        \PY{k+kn}{from} \PY{n+nn}{keras}\PY{n+nn}{.}\PY{n+nn}{models} \PY{k}{import} \PY{n}{Sequential}
        \PY{k+kn}{from} \PY{n+nn}{keras}\PY{n+nn}{.}\PY{n+nn}{layers} \PY{k}{import} \PY{n}{Dense}
        \PY{k+kn}{from} \PY{n+nn}{keras}\PY{n+nn}{.}\PY{n+nn}{layers} \PY{k}{import} \PY{n}{LSTM}
        \PY{k+kn}{import} \PY{n+nn}{keras}
        
        \PY{c+c1}{\PYZsh{} given \PYZhy{} fix random seed \PYZhy{} so we can all reproduce the same results on our default time series}
        \PY{n}{np}\PY{o}{.}\PY{n}{random}\PY{o}{.}\PY{n}{seed}\PY{p}{(}\PY{l+m+mi}{0}\PY{p}{)}
        
        \PY{c+c1}{\PYZsh{} TODO: implement build\PYZus{}part1\PYZus{}RNN in my\PYZus{}answers.py}
        \PY{k+kn}{from} \PY{n+nn}{my\PYZus{}answers} \PY{k}{import} \PY{n}{build\PYZus{}part1\PYZus{}RNN}
        \PY{n}{model} \PY{o}{=} \PY{n}{build\PYZus{}part1\PYZus{}RNN}\PY{p}{(}\PY{n}{window\PYZus{}size}\PY{p}{)}
        \PY{n}{model}\PY{o}{.}\PY{n}{summary}\PY{p}{(}\PY{p}{)}
        
        \PY{c+c1}{\PYZsh{} build model using keras documentation recommended optimizer initialization}
        \PY{n}{optimizer} \PY{o}{=} \PY{n}{keras}\PY{o}{.}\PY{n}{optimizers}\PY{o}{.}\PY{n}{RMSprop}\PY{p}{(}\PY{n}{lr}\PY{o}{=}\PY{l+m+mf}{0.001}\PY{p}{,} \PY{n}{rho}\PY{o}{=}\PY{l+m+mf}{0.9}\PY{p}{,} \PY{n}{epsilon}\PY{o}{=}\PY{l+m+mf}{1e\PYZhy{}08}\PY{p}{,} \PY{n}{decay}\PY{o}{=}\PY{l+m+mf}{0.0}\PY{p}{)}
        
        \PY{c+c1}{\PYZsh{} compile the model}
        \PY{n}{model}\PY{o}{.}\PY{n}{compile}\PY{p}{(}\PY{n}{loss}\PY{o}{=}\PY{l+s+s1}{\PYZsq{}}\PY{l+s+s1}{mean\PYZus{}squared\PYZus{}error}\PY{l+s+s1}{\PYZsq{}}\PY{p}{,} \PY{n}{optimizer}\PY{o}{=}\PY{n}{optimizer}\PY{p}{)}
\end{Verbatim}


    \begin{Verbatim}[commandchars=\\\{\}]
\_\_\_\_\_\_\_\_\_\_\_\_\_\_\_\_\_\_\_\_\_\_\_\_\_\_\_\_\_\_\_\_\_\_\_\_\_\_\_\_\_\_\_\_\_\_\_\_\_\_\_\_\_\_\_\_\_\_\_\_\_\_\_\_\_
Layer (type)                 Output Shape              Param \#   
=================================================================
lstm\_1 (LSTM)                (None, 5)                 140       
\_\_\_\_\_\_\_\_\_\_\_\_\_\_\_\_\_\_\_\_\_\_\_\_\_\_\_\_\_\_\_\_\_\_\_\_\_\_\_\_\_\_\_\_\_\_\_\_\_\_\_\_\_\_\_\_\_\_\_\_\_\_\_\_\_
dense\_1 (Dense)              (None, 1)                 6         
=================================================================
Total params: 146
Trainable params: 146
Non-trainable params: 0
\_\_\_\_\_\_\_\_\_\_\_\_\_\_\_\_\_\_\_\_\_\_\_\_\_\_\_\_\_\_\_\_\_\_\_\_\_\_\_\_\_\_\_\_\_\_\_\_\_\_\_\_\_\_\_\_\_\_\_\_\_\_\_\_\_

    \end{Verbatim}

    With your model built you can now fit the model by activating the cell
below! Note: the number of epochs (np\_epochs) and batch\_size are
preset (so we can all produce the same results). You can choose to
toggle the verbose parameter - which gives you regular updates on the
progress of the algorithm - on and off by setting it to 1 or 0
respectively.

    \begin{Verbatim}[commandchars=\\\{\}]
{\color{incolor}In [{\color{incolor}9}]:} \PY{c+c1}{\PYZsh{} run your model!}
        \PY{n}{model}\PY{o}{.}\PY{n}{fit}\PY{p}{(}\PY{n}{X\PYZus{}train}\PY{p}{,} \PY{n}{y\PYZus{}train}\PY{p}{,} \PY{n}{epochs}\PY{o}{=}\PY{l+m+mi}{1000}\PY{p}{,} \PY{n}{batch\PYZus{}size}\PY{o}{=}\PY{l+m+mi}{25}\PY{p}{,} \PY{n}{verbose}\PY{o}{=}\PY{l+m+mi}{0}\PY{p}{)}
\end{Verbatim}


\begin{Verbatim}[commandchars=\\\{\}]
{\color{outcolor}Out[{\color{outcolor}9}]:} <keras.callbacks.History at 0x22f4c85eb00>
\end{Verbatim}
            
    \subsection{1.5 Checking model
performance}\label{checking-model-performance}

With your model fit we can now make predictions on both our training and
testing sets.

    \begin{Verbatim}[commandchars=\\\{\}]
{\color{incolor}In [{\color{incolor}10}]:} \PY{c+c1}{\PYZsh{} generate predictions for training}
         \PY{n}{train\PYZus{}predict} \PY{o}{=} \PY{n}{model}\PY{o}{.}\PY{n}{predict}\PY{p}{(}\PY{n}{X\PYZus{}train}\PY{p}{)}
         \PY{n}{test\PYZus{}predict} \PY{o}{=} \PY{n}{model}\PY{o}{.}\PY{n}{predict}\PY{p}{(}\PY{n}{X\PYZus{}test}\PY{p}{)}
\end{Verbatim}


    In the next cell we compute training and testing errors using our
trained model - you should be able to achieve at least

\emph{training\_error} \textless{} 0.02

and

\emph{testing\_error} \textless{} 0.02

with your fully trained model.

If either or both of your accuracies are larger than 0.02 re-train your
model - increasing the number of epochs you take (a maximum of around
1,000 should do the job) and/or adjusting your batch\_size.

    \begin{Verbatim}[commandchars=\\\{\}]
{\color{incolor}In [{\color{incolor}11}]:} \PY{c+c1}{\PYZsh{} print out training and testing errors}
         \PY{n}{training\PYZus{}error} \PY{o}{=} \PY{n}{model}\PY{o}{.}\PY{n}{evaluate}\PY{p}{(}\PY{n}{X\PYZus{}train}\PY{p}{,} \PY{n}{y\PYZus{}train}\PY{p}{,} \PY{n}{verbose}\PY{o}{=}\PY{l+m+mi}{0}\PY{p}{)}
         \PY{n+nb}{print}\PY{p}{(}\PY{l+s+s1}{\PYZsq{}}\PY{l+s+s1}{training error = }\PY{l+s+s1}{\PYZsq{}} \PY{o}{+} \PY{n+nb}{str}\PY{p}{(}\PY{n}{training\PYZus{}error}\PY{p}{)}\PY{p}{)}
         
         \PY{n}{testing\PYZus{}error} \PY{o}{=} \PY{n}{model}\PY{o}{.}\PY{n}{evaluate}\PY{p}{(}\PY{n}{X\PYZus{}test}\PY{p}{,} \PY{n}{y\PYZus{}test}\PY{p}{,} \PY{n}{verbose}\PY{o}{=}\PY{l+m+mi}{0}\PY{p}{)}
         \PY{n+nb}{print}\PY{p}{(}\PY{l+s+s1}{\PYZsq{}}\PY{l+s+s1}{testing error = }\PY{l+s+s1}{\PYZsq{}} \PY{o}{+} \PY{n+nb}{str}\PY{p}{(}\PY{n}{testing\PYZus{}error}\PY{p}{)}\PY{p}{)}
\end{Verbatim}


    \begin{Verbatim}[commandchars=\\\{\}]
training error = 0.01582987013865601
testing error = 0.014047877819732178

    \end{Verbatim}

    Activating the next cell plots the original data, as well as both
predictions on the training and testing sets.

    \begin{Verbatim}[commandchars=\\\{\}]
{\color{incolor}In [{\color{incolor}12}]:} \PY{c+c1}{\PYZsh{}\PYZsh{}\PYZsh{} Plot everything \PYZhy{} the original series as well as predictions on training and testing sets}
         \PY{k+kn}{import} \PY{n+nn}{matplotlib}\PY{n+nn}{.}\PY{n+nn}{pyplot} \PY{k}{as} \PY{n+nn}{plt}
         \PY{o}{\PYZpc{}}\PY{k}{matplotlib} inline
         
         \PY{c+c1}{\PYZsh{} plot original series}
         \PY{n}{plt}\PY{o}{.}\PY{n}{plot}\PY{p}{(}\PY{n}{dataset}\PY{p}{,}\PY{n}{color} \PY{o}{=} \PY{l+s+s1}{\PYZsq{}}\PY{l+s+s1}{k}\PY{l+s+s1}{\PYZsq{}}\PY{p}{)}
         
         \PY{c+c1}{\PYZsh{} plot training set prediction}
         \PY{n}{split\PYZus{}pt} \PY{o}{=} \PY{n}{train\PYZus{}test\PYZus{}split} \PY{o}{+} \PY{n}{window\PYZus{}size} 
         \PY{n}{plt}\PY{o}{.}\PY{n}{plot}\PY{p}{(}\PY{n}{np}\PY{o}{.}\PY{n}{arange}\PY{p}{(}\PY{n}{window\PYZus{}size}\PY{p}{,}\PY{n}{split\PYZus{}pt}\PY{p}{,}\PY{l+m+mi}{1}\PY{p}{)}\PY{p}{,}\PY{n}{train\PYZus{}predict}\PY{p}{,}\PY{n}{color} \PY{o}{=} \PY{l+s+s1}{\PYZsq{}}\PY{l+s+s1}{b}\PY{l+s+s1}{\PYZsq{}}\PY{p}{)}
         
         \PY{c+c1}{\PYZsh{} plot testing set prediction}
         \PY{n}{plt}\PY{o}{.}\PY{n}{plot}\PY{p}{(}\PY{n}{np}\PY{o}{.}\PY{n}{arange}\PY{p}{(}\PY{n}{split\PYZus{}pt}\PY{p}{,}\PY{n}{split\PYZus{}pt} \PY{o}{+} \PY{n+nb}{len}\PY{p}{(}\PY{n}{test\PYZus{}predict}\PY{p}{)}\PY{p}{,}\PY{l+m+mi}{1}\PY{p}{)}\PY{p}{,}\PY{n}{test\PYZus{}predict}\PY{p}{,}\PY{n}{color} \PY{o}{=} \PY{l+s+s1}{\PYZsq{}}\PY{l+s+s1}{r}\PY{l+s+s1}{\PYZsq{}}\PY{p}{)}
         
         \PY{c+c1}{\PYZsh{} pretty up graph}
         \PY{n}{plt}\PY{o}{.}\PY{n}{xlabel}\PY{p}{(}\PY{l+s+s1}{\PYZsq{}}\PY{l+s+s1}{day}\PY{l+s+s1}{\PYZsq{}}\PY{p}{)}
         \PY{n}{plt}\PY{o}{.}\PY{n}{ylabel}\PY{p}{(}\PY{l+s+s1}{\PYZsq{}}\PY{l+s+s1}{(normalized) price of Apple stock}\PY{l+s+s1}{\PYZsq{}}\PY{p}{)}
         \PY{n}{plt}\PY{o}{.}\PY{n}{legend}\PY{p}{(}\PY{p}{[}\PY{l+s+s1}{\PYZsq{}}\PY{l+s+s1}{original series}\PY{l+s+s1}{\PYZsq{}}\PY{p}{,}\PY{l+s+s1}{\PYZsq{}}\PY{l+s+s1}{training fit}\PY{l+s+s1}{\PYZsq{}}\PY{p}{,}\PY{l+s+s1}{\PYZsq{}}\PY{l+s+s1}{testing fit}\PY{l+s+s1}{\PYZsq{}}\PY{p}{]}\PY{p}{,}\PY{n}{loc}\PY{o}{=}\PY{l+s+s1}{\PYZsq{}}\PY{l+s+s1}{center left}\PY{l+s+s1}{\PYZsq{}}\PY{p}{,} \PY{n}{bbox\PYZus{}to\PYZus{}anchor}\PY{o}{=}\PY{p}{(}\PY{l+m+mi}{1}\PY{p}{,} \PY{l+m+mf}{0.5}\PY{p}{)}\PY{p}{)}
         \PY{n}{plt}\PY{o}{.}\PY{n}{show}\PY{p}{(}\PY{p}{)}
\end{Verbatim}


    \begin{center}
    \adjustimage{max size={0.9\linewidth}{0.9\paperheight}}{output_28_0.png}
    \end{center}
    { \hspace*{\fill} \\}
    
    \textbf{Note:} you can try out any time series for this exercise! If you
would like to try another see e.g.,
\href{https://datamarket.com/data/list/?q=provider\%3Atsdl}{this site
containing thousands of time series} and pick another one!

    \section{Problem 2: Create a sequence
generator}\label{problem-2-create-a-sequence-generator}

    \subsection{2.1 Getting started}\label{getting-started}

In this project you will implement a popular Recurrent Neural Network
(RNN) architecture to create an English language sequence generator
capable of building semi-coherent English sentences from scratch by
building them up character-by-character. This will require a substantial
amount amount of parameter tuning on a large training corpus (at least
100,000 characters long). In particular for this project we will be
using a complete version of Sir Arthur Conan Doyle's classic book The
Adventures of Sherlock Holmes.

How can we train a machine learning model to generate text
automatically, character-by-character? \emph{By showing the model many
training examples so it can learn a pattern between input and output.}
With this type of text generation each input is a string of valid
characters like this one

\emph{dogs are grea}

while the corresponding output is the next character in the sentence -
which here is 't' (since the complete sentence is 'dogs are great'). We
need to show a model many such examples in order for it to make
reasonable predictions.

\textbf{Fun note:} For those interested in how text generation is being
used check out some of the following fun resources:

\begin{itemize}
\item
  \href{http://www.cs.toronto.edu/~ilya/rnn.html}{Generate wacky
  sentences} with this academic RNN text generator
\item
  Various twitter bots that tweet automatically generated text like
  \href{http://tweet-generator-alex.herokuapp.com/}{this one}.
\item
  the \href{https://github.com/NaNoGenMo/2016}{NanoGenMo} annual contest
  to automatically produce a 50,000+ novel automatically
\item
  \href{https://github.com/genekogan/RobotShakespeare}{Robot
  Shakespeare} a text generator that automatically produces
  Shakespear-esk sentences
\end{itemize}

    \subsection{2.2 Preprocessing a text
dataset}\label{preprocessing-a-text-dataset}

Our first task is to get a large text corpus for use in training, and on
it we perform a several light pre-processing tasks. The default corpus
we will use is the classic book Sherlock Holmes, but you can use a
variety of others as well - so long as they are fairly large (around
100,000 characters or more).

    \begin{Verbatim}[commandchars=\\\{\}]
{\color{incolor}In [{\color{incolor}13}]:} \PY{c+c1}{\PYZsh{} read in the text, transforming everything to lower case}
         \PY{n}{text} \PY{o}{=} \PY{n+nb}{open}\PY{p}{(}\PY{l+s+s1}{\PYZsq{}}\PY{l+s+s1}{datasets/holmes.txt}\PY{l+s+s1}{\PYZsq{}}\PY{p}{)}\PY{o}{.}\PY{n}{read}\PY{p}{(}\PY{p}{)}\PY{o}{.}\PY{n}{lower}\PY{p}{(}\PY{p}{)}
         \PY{n+nb}{print}\PY{p}{(}\PY{l+s+s1}{\PYZsq{}}\PY{l+s+s1}{our original text has }\PY{l+s+s1}{\PYZsq{}} \PY{o}{+} \PY{n+nb}{str}\PY{p}{(}\PY{n+nb}{len}\PY{p}{(}\PY{n}{text}\PY{p}{)}\PY{p}{)} \PY{o}{+} \PY{l+s+s1}{\PYZsq{}}\PY{l+s+s1}{ characters}\PY{l+s+s1}{\PYZsq{}}\PY{p}{)}
\end{Verbatim}


    \begin{Verbatim}[commandchars=\\\{\}]
our original text has 581881 characters

    \end{Verbatim}

    Next, lets examine a bit of the raw text. Because we are interested in
creating sentences of English words automatically by building up each
word character-by-character, we only want to train on valid English
words. In other words - we need to remove all of the other characters
that are not part of English words.

    \begin{Verbatim}[commandchars=\\\{\}]
{\color{incolor}In [{\color{incolor}14}]:} \PY{c+c1}{\PYZsh{}\PYZsh{}\PYZsh{} print out the first 1000 characters of the raw text to get a sense of what we need to throw out}
         \PY{n}{text}\PY{p}{[}\PY{p}{:}\PY{l+m+mi}{2000}\PY{p}{]}
\end{Verbatim}


\begin{Verbatim}[commandchars=\\\{\}]
{\color{outcolor}Out[{\color{outcolor}14}]:} "project gutenberg's the adventures of sherlock holmes, by arthur conan doyle\textbackslash{}n\textbackslash{}nthis ebook is for the use of anyone anywhere at no cost and with\textbackslash{}nalmost no restrictions whatsoever.  you may copy it, give it away or\textbackslash{}nre-use it under the terms of the project gutenberg license included\textbackslash{}nwith this ebook or online at www.gutenberg.net\textbackslash{}n\textbackslash{}n\textbackslash{}ntitle: the adventures of sherlock holmes\textbackslash{}n\textbackslash{}nauthor: arthur conan doyle\textbackslash{}n\textbackslash{}nposting date: april 18, 2011 [ebook \#1661]\textbackslash{}nfirst posted: november 29, 2002\textbackslash{}n\textbackslash{}nlanguage: english\textbackslash{}n\textbackslash{}n\textbackslash{}n*** start of this project gutenberg ebook the adventures of sherlock holmes ***\textbackslash{}n\textbackslash{}n\textbackslash{}n\textbackslash{}n\textbackslash{}nproduced by an anonymous project gutenberg volunteer and jose menendez\textbackslash{}n\textbackslash{}n\textbackslash{}n\textbackslash{}n\textbackslash{}n\textbackslash{}n\textbackslash{}n\textbackslash{}n\textbackslash{}n\textbackslash{}nthe adventures of sherlock holmes\textbackslash{}n\textbackslash{}nby\textbackslash{}n\textbackslash{}nsir arthur conan doyle\textbackslash{}n\textbackslash{}n\textbackslash{}n\textbackslash{}n   i. a scandal in bohemia\textbackslash{}n  ii. the red-headed league\textbackslash{}n iii. a case of identity\textbackslash{}n  iv. the boscombe valley mystery\textbackslash{}n   v. the five orange pips\textbackslash{}n  vi. the man with the twisted lip\textbackslash{}n vii. the adventure of the blue carbuncle\textbackslash{}nviii. the adventure of the speckled band\textbackslash{}n  ix. the adventure of the engineer's thumb\textbackslash{}n   x. the adventure of the noble bachelor\textbackslash{}n  xi. the adventure of the beryl coronet\textbackslash{}n xii. the adventure of the copper beeches\textbackslash{}n\textbackslash{}n\textbackslash{}n\textbackslash{}n\textbackslash{}nadventure i. a scandal in bohemia\textbackslash{}n\textbackslash{}ni.\textbackslash{}n\textbackslash{}nto sherlock holmes she is always the woman. i have seldom heard\textbackslash{}nhim mention her under any other name. in his eyes she eclipses\textbackslash{}nand predominates the whole of her sex. it was not that he felt\textbackslash{}nany emotion akin to love for irene adler. all emotions, and that\textbackslash{}none particularly, were abhorrent to his cold, precise but\textbackslash{}nadmirably balanced mind. he was, i take it, the most perfect\textbackslash{}nreasoning and observing machine that the world has seen, but as a\textbackslash{}nlover he would have placed himself in a false position. he never\textbackslash{}nspoke of the softer passions, save with a gibe and a sneer. they\textbackslash{}nwere admirable things for the observer--excellent for drawing the\textbackslash{}nveil from men's motives and actions. but for the trained reasoner\textbackslash{}nto admit such intrusions into his own delicate and finely\textbackslash{}nadjusted temperament was to introduce a di"
\end{Verbatim}
            
    Wow - there's a lot of junk here (i.e., weird uncommon character
combinations - as this first character chunk contains the title and
author page, as well as table of contents)! To keep things simple, we
want to train our RNN on a large chunk of more typical English sentences
- we don't want it to start thinking non-english words or strange
characters are valid! - so lets clean up the data a bit.

First, since the dataset is so large and the first few hundred
characters contain a lot of junk, lets cut it out. Lets also
find-and-replace those newline tags with empty spaces.

    \begin{Verbatim}[commandchars=\\\{\}]
{\color{incolor}In [{\color{incolor}15}]:} \PY{c+c1}{\PYZsh{}\PYZsh{}\PYZsh{} find and replace \PYZsq{}\PYZbs{}n\PYZsq{} and \PYZsq{}\PYZbs{}r\PYZsq{} symbols \PYZhy{} replacing them }
         \PY{n}{text} \PY{o}{=} \PY{n}{text}\PY{p}{[}\PY{l+m+mi}{1302}\PY{p}{:}\PY{p}{]}
         \PY{n}{text} \PY{o}{=} \PY{n}{text}\PY{o}{.}\PY{n}{replace}\PY{p}{(}\PY{l+s+s1}{\PYZsq{}}\PY{l+s+se}{\PYZbs{}n}\PY{l+s+s1}{\PYZsq{}}\PY{p}{,}\PY{l+s+s1}{\PYZsq{}}\PY{l+s+s1}{ }\PY{l+s+s1}{\PYZsq{}}\PY{p}{)}    \PY{c+c1}{\PYZsh{} replacing \PYZsq{}\PYZbs{}n\PYZsq{} with \PYZsq{}\PYZsq{} simply removes the sequence}
         \PY{n}{text} \PY{o}{=} \PY{n}{text}\PY{o}{.}\PY{n}{replace}\PY{p}{(}\PY{l+s+s1}{\PYZsq{}}\PY{l+s+se}{\PYZbs{}r}\PY{l+s+s1}{\PYZsq{}}\PY{p}{,}\PY{l+s+s1}{\PYZsq{}}\PY{l+s+s1}{ }\PY{l+s+s1}{\PYZsq{}}\PY{p}{)}
\end{Verbatim}


    Lets see how the first 1000 characters of our text looks now!

    \begin{Verbatim}[commandchars=\\\{\}]
{\color{incolor}In [{\color{incolor}16}]:} \PY{c+c1}{\PYZsh{}\PYZsh{}\PYZsh{} print out the first 1000 characters of the raw text to get a sense of what we need to throw out}
         \PY{n}{text}\PY{p}{[}\PY{p}{:}\PY{l+m+mi}{1000}\PY{p}{]}
\end{Verbatim}


\begin{Verbatim}[commandchars=\\\{\}]
{\color{outcolor}Out[{\color{outcolor}16}]:} " his eyes she eclipses and predominates the whole of her sex. it was not that he felt any emotion akin to love for irene adler. all emotions, and that one particularly, were abhorrent to his cold, precise but admirably balanced mind. he was, i take it, the most perfect reasoning and observing machine that the world has seen, but as a lover he would have placed himself in a false position. he never spoke of the softer passions, save with a gibe and a sneer. they were admirable things for the observer--excellent for drawing the veil from men's motives and actions. but for the trained reasoner to admit such intrusions into his own delicate and finely adjusted temperament was to introduce a distracting factor which might throw a doubt upon all his mental results. grit in a sensitive instrument, or a crack in one of his own high-power lenses, would not be more disturbing than a strong emotion in a nature such as his. and yet there was but one woman to him, and that woman was the late irene "
\end{Verbatim}
            
    \paragraph{TODO: finish cleaning the
text}\label{todo-finish-cleaning-the-text}

Lets make sure we haven't left any other atypical characters (commas,
periods, etc., are ok) lurking around in the depths of the text. You can
do this by enumerating all the text's unique characters, examining them,
and then replacing any unwanted characters with empty spaces! Once we
find all of the text's unique characters, we can remove all of the
atypical ones in the next cell. Note: don't remove the punctuation marks
given in my\_answers.py.

    \begin{Verbatim}[commandchars=\\\{\}]
{\color{incolor}In [{\color{incolor}17}]:} \PY{c+c1}{\PYZsh{}\PYZsh{}\PYZsh{} TODO: implement cleaned\PYZus{}text in my\PYZus{}answers.py}
         \PY{k+kn}{from} \PY{n+nn}{my\PYZus{}answers} \PY{k}{import} \PY{n}{cleaned\PYZus{}text}
         
         \PY{n}{text} \PY{o}{=} \PY{n}{cleaned\PYZus{}text}\PY{p}{(}\PY{n}{text}\PY{p}{)}
         
         \PY{c+c1}{\PYZsh{} shorten any extra dead space created above}
         \PY{n}{text} \PY{o}{=} \PY{n}{text}\PY{o}{.}\PY{n}{replace}\PY{p}{(}\PY{l+s+s1}{\PYZsq{}}\PY{l+s+s1}{  }\PY{l+s+s1}{\PYZsq{}}\PY{p}{,}\PY{l+s+s1}{\PYZsq{}}\PY{l+s+s1}{ }\PY{l+s+s1}{\PYZsq{}}\PY{p}{)}
\end{Verbatim}


    With your chosen characters removed print out the first few hundred
lines again just to double check that everything looks good.

    \begin{Verbatim}[commandchars=\\\{\}]
{\color{incolor}In [{\color{incolor}18}]:} \PY{c+c1}{\PYZsh{}\PYZsh{}\PYZsh{} print out the first 2000 characters of the raw text to get a sense of what we need to throw out}
         \PY{n}{text}\PY{p}{[}\PY{p}{:}\PY{l+m+mi}{2000}\PY{p}{]}
\end{Verbatim}


\begin{Verbatim}[commandchars=\\\{\}]
{\color{outcolor}Out[{\color{outcolor}18}]:} ' his eyes she eclipses and predominates the whole of her sex. it was not that he felt any emotion akin to love for irene adler. all emotions, and that one particularly, were abhorrent to his cold, precise but admirably balanced mind. he was, i take it, the most perfect reasoning and observing machine that the world has seen, but as a lover he would have placed himself in a false position. he never spoke of the softer passions, save with a gibe and a sneer. they were admirable things for the observer excellent for drawing the veil from men s motives and actions. but for the trained reasoner to admit such intrusions into his own delicate and finely adjusted temperament was to introduce a distracting factor which might throw a doubt upon all his mental results. grit in a sensitive instrument, or a crack in one of his own high power lenses, would not be more disturbing than a strong emotion in a nature such as his. and yet there was but one woman to him, and that woman was the late irene adler, of dubious and questionable memory. i had seen little of holmes lately. my marriage had drifted us away from each other. my own complete happiness, and the home centred interests which rise up around the man who first finds himself master of his own establishment, were sufficient to absorb all my attention, while holmes, who loathed every form of society with his whole bohemian soul, remained in our lodgings in baker street, buried among his old books, and alternating from week to week between cocaine and ambition, the drowsiness of the drug, and the fierce energy of his own keen nature. he was still, as ever, deeply attracted by the study of crime, and occupied his immense faculties and extraordinary powers of observation in following out those clues, and clearing up those mysteries which had been abandoned as hopeless by the official police. from time to time i heard some vague account of his doings: of his summons to odessa in the case of the trepoff murder, of his clearing up'
\end{Verbatim}
            
    Now that we have thrown out a good number of non-English
characters/character sequences lets print out some statistics about the
dataset - including number of total characters and number of unique
characters.

    \begin{Verbatim}[commandchars=\\\{\}]
{\color{incolor}In [{\color{incolor}19}]:} \PY{c+c1}{\PYZsh{} count the number of unique characters in the text}
         \PY{n}{chars} \PY{o}{=} \PY{n+nb}{sorted}\PY{p}{(}\PY{n+nb}{list}\PY{p}{(}\PY{n+nb}{set}\PY{p}{(}\PY{n}{text}\PY{p}{)}\PY{p}{)}\PY{p}{)}
         
         \PY{c+c1}{\PYZsh{} print some of the text, as well as statistics}
         \PY{n+nb}{print} \PY{p}{(}\PY{l+s+s2}{\PYZdq{}}\PY{l+s+s2}{this corpus has }\PY{l+s+s2}{\PYZdq{}} \PY{o}{+}  \PY{n+nb}{str}\PY{p}{(}\PY{n+nb}{len}\PY{p}{(}\PY{n}{text}\PY{p}{)}\PY{p}{)} \PY{o}{+} \PY{l+s+s2}{\PYZdq{}}\PY{l+s+s2}{ total number of characters}\PY{l+s+s2}{\PYZdq{}}\PY{p}{)}
         \PY{n+nb}{print} \PY{p}{(}\PY{l+s+s2}{\PYZdq{}}\PY{l+s+s2}{this corpus has }\PY{l+s+s2}{\PYZdq{}} \PY{o}{+}  \PY{n+nb}{str}\PY{p}{(}\PY{n+nb}{len}\PY{p}{(}\PY{n}{chars}\PY{p}{)}\PY{p}{)} \PY{o}{+} \PY{l+s+s2}{\PYZdq{}}\PY{l+s+s2}{ unique characters}\PY{l+s+s2}{\PYZdq{}}\PY{p}{)}
\end{Verbatim}


    \begin{Verbatim}[commandchars=\\\{\}]
this corpus has 573688 total number of characters
this corpus has 33 unique characters

    \end{Verbatim}

    \subsection{2.3 Cutting data into input/output
pairs}\label{cutting-data-into-inputoutput-pairs}

Now that we have our text all cleaned up, how can we use it to train a
model to generate sentences automatically? First we need to train a
machine learning model - and in order to do that we need a set of
input/output pairs for a model to train on. How can we create a set of
input/output pairs from our text to train on?

Remember in part 1 of this notebook how we used a sliding window to
extract input/output pairs from a time series? We do the same thing
here! We slide a window of length \(T\) along our giant text corpus -
everything in the window becomes one input while the character following
becomes its corresponding output. This process of extracting
input/output pairs is illustrated in the gif below on a small example
text using a window size of T = 5.

Notice one aspect of the sliding window in this gif that does not mirror
the analogous gif for time series shown in part 1 of the notebook - we
do not need to slide the window along one character at a time but can
move by a fixed step size \(M\) greater than 1 (in the gif indeed
\(M = 1\)). This is done with large input texts (like ours which has
over 500,000 characters!) when sliding the window along one character at
a time we would create far too many input/output pairs to be able to
reasonably compute with.

More formally lets denote our text corpus - which is one long string of
characters - as follows

\[s_{0},s_{1},s_{2},...,s_{P}\]

where \(P\) is the length of the text (again for our text
\(P \approx 500,000!\)). Sliding a window of size T = 5 with a step
length of M = 1 (these are the parameters shown in the gif above) over
this sequence produces the following list of input/output pairs

\[\begin{array}{c|c}
\text{Input} & \text{Output}\\
\hline \color{CornflowerBlue} {\langle s_{1},s_{2},s_{3},s_{4},s_{5}\rangle} & \color{Goldenrod}{ s_{6}} \\
\ \color{CornflowerBlue} {\langle s_{2},s_{3},s_{4},s_{5},s_{6} \rangle } & \color{Goldenrod} {s_{7} } \\
\color{CornflowerBlue}  {\vdots} & \color{Goldenrod} {\vdots}\\
\color{CornflowerBlue} { \langle s_{P-5},s_{P-4},s_{P-3},s_{P-2},s_{P-1} \rangle } & \color{Goldenrod} {s_{P}}
\end{array}\]

Notice here that each input is a sequence (or vector) of 5 characters
(and in general has length equal to the window size T) while each
corresponding output is a single character. We created around P total
number of input/output pairs (for general step size M we create around
ceil(P/M) pairs).

    Now its time for you to window the input time series as described above!

\textbf{TODO:} Create a function that runs a sliding window along the
input text and creates associated input/output pairs. A skeleton
function has been provided for you. Note that this function should input
a) the text b) the window size and c) the step size, and return the
input/output sequences. Note: the return items should be \emph{lists} -
not numpy arrays.

(remember to copy your completed function into the script
\emph{my\_answers.py} function titled \emph{window\_transform\_text}
before submitting your project)

    \begin{Verbatim}[commandchars=\\\{\}]
{\color{incolor}In [{\color{incolor}20}]:} \PY{c+c1}{\PYZsh{}\PYZsh{}\PYZsh{} TODO: implement window\PYZus{}transform\PYZus{}series in my\PYZus{}answers.py}
         \PY{k+kn}{from} \PY{n+nn}{my\PYZus{}answers} \PY{k}{import} \PY{n}{window\PYZus{}transform\PYZus{}series}
\end{Verbatim}


    With our function complete we can now use it to produce input/output
pairs! We employ the function in the next cell, where the window\_size =
50 and step\_size = 5.

    \begin{Verbatim}[commandchars=\\\{\}]
{\color{incolor}In [{\color{incolor}21}]:} \PY{c+c1}{\PYZsh{} run your text window\PYZhy{}ing function }
         \PY{n}{window\PYZus{}size} \PY{o}{=} \PY{l+m+mi}{50}
         \PY{n}{step\PYZus{}size} \PY{o}{=} \PY{l+m+mi}{5}
         \PY{n}{inputs}\PY{p}{,} \PY{n}{outputs} \PY{o}{=} \PY{n}{window\PYZus{}transform\PYZus{}text}\PY{p}{(}\PY{n}{text}\PY{p}{,}\PY{n}{window\PYZus{}size}\PY{p}{,}\PY{n}{step\PYZus{}size}\PY{p}{)}
\end{Verbatim}


    Lets print out a few input/output pairs to verify that we have made the
right sort of stuff!

    \begin{Verbatim}[commandchars=\\\{\}]
{\color{incolor}In [{\color{incolor}22}]:} \PY{c+c1}{\PYZsh{} print out a few of the input/output pairs to verify that we\PYZsq{}ve made the right kind of stuff to learn from}
         \PY{n+nb}{print}\PY{p}{(}\PY{l+s+s1}{\PYZsq{}}\PY{l+s+s1}{input = }\PY{l+s+s1}{\PYZsq{}} \PY{o}{+} \PY{n}{inputs}\PY{p}{[}\PY{l+m+mi}{3}\PY{p}{]}\PY{p}{)}
         \PY{n+nb}{print}\PY{p}{(}\PY{l+s+s1}{\PYZsq{}}\PY{l+s+s1}{output = }\PY{l+s+s1}{\PYZsq{}} \PY{o}{+} \PY{n}{outputs}\PY{p}{[}\PY{l+m+mi}{3}\PY{p}{]}\PY{p}{)}
         \PY{n+nb}{print}\PY{p}{(}\PY{l+s+s1}{\PYZsq{}}\PY{l+s+s1}{\PYZhy{}\PYZhy{}\PYZhy{}\PYZhy{}\PYZhy{}\PYZhy{}\PYZhy{}\PYZhy{}\PYZhy{}\PYZhy{}\PYZhy{}\PYZhy{}\PYZhy{}\PYZhy{}}\PY{l+s+s1}{\PYZsq{}}\PY{p}{)}
         \PY{n+nb}{print}\PY{p}{(}\PY{l+s+s1}{\PYZsq{}}\PY{l+s+s1}{input = }\PY{l+s+s1}{\PYZsq{}} \PY{o}{+} \PY{n}{inputs}\PY{p}{[}\PY{l+m+mi}{101}\PY{p}{]}\PY{p}{)}
         \PY{n+nb}{print}\PY{p}{(}\PY{l+s+s1}{\PYZsq{}}\PY{l+s+s1}{output = }\PY{l+s+s1}{\PYZsq{}} \PY{o}{+} \PY{n}{outputs}\PY{p}{[}\PY{l+m+mi}{101}\PY{p}{]}\PY{p}{)}
\end{Verbatim}


    \begin{Verbatim}[commandchars=\\\{\}]
input = clipses and predominates the whole of her sex. it 
output = w
--------------
input = excellent for drawing the veil from men s motives 
output = a

    \end{Verbatim}

    Looks good!

    \subsection{2.4 Wait, what kind of problem is text generation
again?}\label{wait-what-kind-of-problem-is-text-generation-again}

In part 1 of this notebook we used the same pre-processing technique -
the sliding window - to produce a set of training input/output pairs to
tackle the problem of time series prediction \emph{by treating the
problem as one of regression}. So what sort of problem do we have here
now, with text generation? Well, the time series prediction was a
regression problem because the output (one value of the time series) was
a continuous value. Here - for character-by-character text generation -
each output is a \emph{single character}. This isn't a continuous value
- but a distinct class - therefore \textbf{character-by-character text
generation is a classification problem}.

How many classes are there in the data? Well, the number of classes is
equal to the number of unique characters we have to predict! How many of
those were there in our dataset again? Lets print out the value again.

    \begin{Verbatim}[commandchars=\\\{\}]
{\color{incolor}In [{\color{incolor}23}]:} \PY{c+c1}{\PYZsh{} print out the number of unique characters in the dataset}
         \PY{n}{chars} \PY{o}{=} \PY{n+nb}{sorted}\PY{p}{(}\PY{n+nb}{list}\PY{p}{(}\PY{n+nb}{set}\PY{p}{(}\PY{n}{text}\PY{p}{)}\PY{p}{)}\PY{p}{)}
         \PY{n+nb}{print} \PY{p}{(}\PY{l+s+s2}{\PYZdq{}}\PY{l+s+s2}{this corpus has }\PY{l+s+s2}{\PYZdq{}} \PY{o}{+}  \PY{n+nb}{str}\PY{p}{(}\PY{n+nb}{len}\PY{p}{(}\PY{n}{chars}\PY{p}{)}\PY{p}{)} \PY{o}{+} \PY{l+s+s2}{\PYZdq{}}\PY{l+s+s2}{ unique characters}\PY{l+s+s2}{\PYZdq{}}\PY{p}{)}
         \PY{n+nb}{print} \PY{p}{(}\PY{l+s+s1}{\PYZsq{}}\PY{l+s+s1}{and these characters are }\PY{l+s+s1}{\PYZsq{}}\PY{p}{)}
         \PY{n+nb}{print} \PY{p}{(}\PY{n}{chars}\PY{p}{)}
\end{Verbatim}


    \begin{Verbatim}[commandchars=\\\{\}]
this corpus has 33 unique characters
and these characters are 
[' ', '!', ',', '.', ':', ';', '?', 'a', 'b', 'c', 'd', 'e', 'f', 'g', 'h', 'i', 'j', 'k', 'l', 'm', 'n', 'o', 'p', 'q', 'r', 's', 't', 'u', 'v', 'w', 'x', 'y', 'z']

    \end{Verbatim}

    Rockin' - so we have a multiclass classification problem on our hands!

    \subsection{2.5 One-hot encoding
characters}\label{one-hot-encoding-characters}

The last issue we have to deal with is representing our text data as
numerical data so that we can use it as an input to a neural network.
One of the conceptually simplest ways of doing this is via a 'one-hot
encoding' scheme. Here's how it works.

We transform each character in our inputs/outputs into a vector with
length equal to the number of unique characters in our text. This vector
is all zeros except one location where we place a 1 - and this location
is unique to each character type. e.g., we transform 'a', 'b', and 'c'
as follows

\[a\longleftarrow\left[\begin{array}{c}
1\\
0\\
0\\
\vdots\\
0\\
0
\end{array}\right]\,\,\,\,\,\,\,b\longleftarrow\left[\begin{array}{c}
0\\
1\\
0\\
\vdots\\
0\\
0
\end{array}\right]\,\,\,\,\,c\longleftarrow\left[\begin{array}{c}
0\\
0\\
1\\
\vdots\\
0\\
0 
\end{array}\right]\cdots\]

where each vector has 32 entries (or in general: number of entries =
number of unique characters in text).

    The first practical step towards doing this one-hot encoding is to form
a dictionary mapping each unique character to a unique integer, and one
dictionary to do the reverse mapping. We can then use these dictionaries
to quickly make our one-hot encodings, as well as re-translate (from
integers to characters) the results of our trained RNN classification
model.

    \begin{Verbatim}[commandchars=\\\{\}]
{\color{incolor}In [{\color{incolor}24}]:} \PY{c+c1}{\PYZsh{} this dictionary is a function mapping each unique character to a unique integer}
         \PY{n}{chars\PYZus{}to\PYZus{}indices} \PY{o}{=} \PY{n+nb}{dict}\PY{p}{(}\PY{p}{(}\PY{n}{c}\PY{p}{,} \PY{n}{i}\PY{p}{)} \PY{k}{for} \PY{n}{i}\PY{p}{,} \PY{n}{c} \PY{o+ow}{in} \PY{n+nb}{enumerate}\PY{p}{(}\PY{n}{chars}\PY{p}{)}\PY{p}{)}  \PY{c+c1}{\PYZsh{} map each unique character to unique integer}
         
         \PY{c+c1}{\PYZsh{} this dictionary is a function mapping each unique integer back to a unique character}
         \PY{n}{indices\PYZus{}to\PYZus{}chars} \PY{o}{=} \PY{n+nb}{dict}\PY{p}{(}\PY{p}{(}\PY{n}{i}\PY{p}{,} \PY{n}{c}\PY{p}{)} \PY{k}{for} \PY{n}{i}\PY{p}{,} \PY{n}{c} \PY{o+ow}{in} \PY{n+nb}{enumerate}\PY{p}{(}\PY{n}{chars}\PY{p}{)}\PY{p}{)}  \PY{c+c1}{\PYZsh{} map each unique integer back to unique character}
\end{Verbatim}


    Now we can transform our input/output pairs - consisting of characters -
to equivalent input/output pairs made up of one-hot encoded vectors. In
the next cell we provide a function for doing just this: it takes in the
raw character input/outputs and returns their numerical versions. In
particular the numerical input is given as \(\bf{X}\), and numerical
output is given as the \(\bf{y}\)

    \begin{Verbatim}[commandchars=\\\{\}]
{\color{incolor}In [{\color{incolor}25}]:} \PY{c+c1}{\PYZsh{} transform character\PYZhy{}based input/output into equivalent numerical versions}
         \PY{k}{def} \PY{n+nf}{encode\PYZus{}io\PYZus{}pairs}\PY{p}{(}\PY{n}{text}\PY{p}{,}\PY{n}{window\PYZus{}size}\PY{p}{,}\PY{n}{step\PYZus{}size}\PY{p}{)}\PY{p}{:}
             \PY{c+c1}{\PYZsh{} number of unique chars}
             \PY{n}{chars} \PY{o}{=} \PY{n+nb}{sorted}\PY{p}{(}\PY{n+nb}{list}\PY{p}{(}\PY{n+nb}{set}\PY{p}{(}\PY{n}{text}\PY{p}{)}\PY{p}{)}\PY{p}{)}
             \PY{n}{num\PYZus{}chars} \PY{o}{=} \PY{n+nb}{len}\PY{p}{(}\PY{n}{chars}\PY{p}{)}
             
             \PY{c+c1}{\PYZsh{} cut up text into character input/output pairs}
             \PY{n}{inputs}\PY{p}{,} \PY{n}{outputs} \PY{o}{=} \PY{n}{window\PYZus{}transform\PYZus{}text}\PY{p}{(}\PY{n}{text}\PY{p}{,}\PY{n}{window\PYZus{}size}\PY{p}{,}\PY{n}{step\PYZus{}size}\PY{p}{)}
             
             \PY{c+c1}{\PYZsh{} create empty vessels for one\PYZhy{}hot encoded input/output}
             \PY{n}{X} \PY{o}{=} \PY{n}{np}\PY{o}{.}\PY{n}{zeros}\PY{p}{(}\PY{p}{(}\PY{n+nb}{len}\PY{p}{(}\PY{n}{inputs}\PY{p}{)}\PY{p}{,} \PY{n}{window\PYZus{}size}\PY{p}{,} \PY{n}{num\PYZus{}chars}\PY{p}{)}\PY{p}{,} \PY{n}{dtype}\PY{o}{=}\PY{n}{np}\PY{o}{.}\PY{n}{bool}\PY{p}{)}
             \PY{n}{y} \PY{o}{=} \PY{n}{np}\PY{o}{.}\PY{n}{zeros}\PY{p}{(}\PY{p}{(}\PY{n+nb}{len}\PY{p}{(}\PY{n}{inputs}\PY{p}{)}\PY{p}{,} \PY{n}{num\PYZus{}chars}\PY{p}{)}\PY{p}{,} \PY{n}{dtype}\PY{o}{=}\PY{n}{np}\PY{o}{.}\PY{n}{bool}\PY{p}{)}
             
             \PY{c+c1}{\PYZsh{} loop over inputs/outputs and transform and store in X/y}
             \PY{k}{for} \PY{n}{i}\PY{p}{,} \PY{n}{sentence} \PY{o+ow}{in} \PY{n+nb}{enumerate}\PY{p}{(}\PY{n}{inputs}\PY{p}{)}\PY{p}{:}
                 \PY{k}{for} \PY{n}{t}\PY{p}{,} \PY{n}{char} \PY{o+ow}{in} \PY{n+nb}{enumerate}\PY{p}{(}\PY{n}{sentence}\PY{p}{)}\PY{p}{:}
                     \PY{n}{X}\PY{p}{[}\PY{n}{i}\PY{p}{,} \PY{n}{t}\PY{p}{,} \PY{n}{chars\PYZus{}to\PYZus{}indices}\PY{p}{[}\PY{n}{char}\PY{p}{]}\PY{p}{]} \PY{o}{=} \PY{l+m+mi}{1}
                 \PY{n}{y}\PY{p}{[}\PY{n}{i}\PY{p}{,} \PY{n}{chars\PYZus{}to\PYZus{}indices}\PY{p}{[}\PY{n}{outputs}\PY{p}{[}\PY{n}{i}\PY{p}{]}\PY{p}{]}\PY{p}{]} \PY{o}{=} \PY{l+m+mi}{1}
                 
             \PY{k}{return} \PY{n}{X}\PY{p}{,}\PY{n}{y}
\end{Verbatim}


    Now run the one-hot encoding function by activating the cell below and
transform our input/output pairs!

    \begin{Verbatim}[commandchars=\\\{\}]
{\color{incolor}In [{\color{incolor}26}]:} \PY{c+c1}{\PYZsh{} use your function}
         \PY{n}{window\PYZus{}size} \PY{o}{=} \PY{l+m+mi}{100}
         \PY{n}{step\PYZus{}size} \PY{o}{=} \PY{l+m+mi}{5}
         \PY{n}{X}\PY{p}{,}\PY{n}{y} \PY{o}{=} \PY{n}{encode\PYZus{}io\PYZus{}pairs}\PY{p}{(}\PY{n}{text}\PY{p}{,}\PY{n}{window\PYZus{}size}\PY{p}{,}\PY{n}{step\PYZus{}size}\PY{p}{)}
\end{Verbatim}


    \subsection{2.6 Setting up our RNN}\label{setting-up-our-rnn}

With our dataset loaded and the input/output pairs extracted /
transformed we can now begin setting up our RNN for training. Again we
will use Keras to quickly build a single hidden layer RNN - where our
hidden layer consists of LSTM modules.

Time to get to work: build a 3 layer RNN model of the following
specification

\begin{itemize}
\tightlist
\item
  layer 1 should be an LSTM module with 200 hidden units
  -\/-\textgreater{} note this should have input\_shape =
  (window\_size,len(chars)) where len(chars) = number of unique
  characters in your cleaned text
\item
  layer 2 should be a linear module, fully connected, with len(chars)
  hidden units -\/-\textgreater{} where len(chars) = number of unique
  characters in your cleaned text
\item
  layer 3 should be a softmax activation ( since we are solving a
  \emph{multiclass classification})
\item
  Use the \textbf{categorical\_crossentropy} loss
\end{itemize}

This network can be constructed using just a few lines - as with the RNN
network you made in part 1 of this notebook. See e.g., the
\href{https://keras.io/getting-started/sequential-model-guide/}{general
Keras documentation} and the
\href{https://keras.io/layers/recurrent/}{LSTM documentation in
particular} for examples of how to quickly use Keras to build neural
network models.

    \begin{Verbatim}[commandchars=\\\{\}]
{\color{incolor}In [{\color{incolor}27}]:} \PY{c+c1}{\PYZsh{}\PYZsh{}\PYZsh{} necessary functions from the keras library}
         \PY{k+kn}{from} \PY{n+nn}{keras}\PY{n+nn}{.}\PY{n+nn}{models} \PY{k}{import} \PY{n}{Sequential}
         \PY{k+kn}{from} \PY{n+nn}{keras}\PY{n+nn}{.}\PY{n+nn}{layers} \PY{k}{import} \PY{n}{Dense}\PY{p}{,} \PY{n}{Activation}\PY{p}{,} \PY{n}{LSTM}
         \PY{k+kn}{from} \PY{n+nn}{keras}\PY{n+nn}{.}\PY{n+nn}{optimizers} \PY{k}{import} \PY{n}{RMSprop}
         \PY{k+kn}{from} \PY{n+nn}{keras}\PY{n+nn}{.}\PY{n+nn}{utils}\PY{n+nn}{.}\PY{n+nn}{data\PYZus{}utils} \PY{k}{import} \PY{n}{get\PYZus{}file}
         \PY{k+kn}{import} \PY{n+nn}{keras}
         \PY{k+kn}{import} \PY{n+nn}{random}
         
         \PY{c+c1}{\PYZsh{} TODO implement build\PYZus{}part2\PYZus{}RNN in my\PYZus{}answers.py}
         \PY{k+kn}{from} \PY{n+nn}{my\PYZus{}answers} \PY{k}{import} \PY{n}{build\PYZus{}part2\PYZus{}RNN}
         
         \PY{n}{model} \PY{o}{=} \PY{n}{build\PYZus{}part2\PYZus{}RNN}\PY{p}{(}\PY{n}{window\PYZus{}size}\PY{p}{,} \PY{n+nb}{len}\PY{p}{(}\PY{n}{chars}\PY{p}{)}\PY{p}{)}
         \PY{n}{model}\PY{o}{.}\PY{n}{summary}\PY{p}{(}\PY{p}{)}
         
         \PY{c+c1}{\PYZsh{} initialize optimizer}
         \PY{c+c1}{\PYZsh{} CHANGED learning rate to 0.01 from 0.001}
         \PY{n}{optimizer} \PY{o}{=} \PY{n}{keras}\PY{o}{.}\PY{n}{optimizers}\PY{o}{.}\PY{n}{RMSprop}\PY{p}{(}\PY{n}{lr}\PY{o}{=}\PY{l+m+mf}{0.01}\PY{p}{,} \PY{n}{rho}\PY{o}{=}\PY{l+m+mf}{0.9}\PY{p}{,} \PY{n}{epsilon}\PY{o}{=}\PY{l+m+mf}{1e\PYZhy{}08}\PY{p}{,} \PY{n}{decay}\PY{o}{=}\PY{l+m+mf}{0.0}\PY{p}{)}
         
         \PY{c+c1}{\PYZsh{} compile model \PYZhy{}\PYZhy{}\PYZgt{} make sure initialized optimizer and callbacks \PYZhy{} as defined above \PYZhy{} are used}
         \PY{n}{model}\PY{o}{.}\PY{n}{compile}\PY{p}{(}\PY{n}{loss}\PY{o}{=}\PY{l+s+s1}{\PYZsq{}}\PY{l+s+s1}{categorical\PYZus{}crossentropy}\PY{l+s+s1}{\PYZsq{}}\PY{p}{,} \PY{n}{optimizer}\PY{o}{=}\PY{n}{optimizer}\PY{p}{)}
\end{Verbatim}


    \begin{Verbatim}[commandchars=\\\{\}]
\_\_\_\_\_\_\_\_\_\_\_\_\_\_\_\_\_\_\_\_\_\_\_\_\_\_\_\_\_\_\_\_\_\_\_\_\_\_\_\_\_\_\_\_\_\_\_\_\_\_\_\_\_\_\_\_\_\_\_\_\_\_\_\_\_
Layer (type)                 Output Shape              Param \#   
=================================================================
lstm\_2 (LSTM)                (None, 200)               187200    
\_\_\_\_\_\_\_\_\_\_\_\_\_\_\_\_\_\_\_\_\_\_\_\_\_\_\_\_\_\_\_\_\_\_\_\_\_\_\_\_\_\_\_\_\_\_\_\_\_\_\_\_\_\_\_\_\_\_\_\_\_\_\_\_\_
dense\_2 (Dense)              (None, 33)                6633      
\_\_\_\_\_\_\_\_\_\_\_\_\_\_\_\_\_\_\_\_\_\_\_\_\_\_\_\_\_\_\_\_\_\_\_\_\_\_\_\_\_\_\_\_\_\_\_\_\_\_\_\_\_\_\_\_\_\_\_\_\_\_\_\_\_
activation\_1 (Activation)    (None, 33)                0         
=================================================================
Total params: 193,833
Trainable params: 193,833
Non-trainable params: 0
\_\_\_\_\_\_\_\_\_\_\_\_\_\_\_\_\_\_\_\_\_\_\_\_\_\_\_\_\_\_\_\_\_\_\_\_\_\_\_\_\_\_\_\_\_\_\_\_\_\_\_\_\_\_\_\_\_\_\_\_\_\_\_\_\_

    \end{Verbatim}

    \subsection{2.7 Training our RNN model for text
generation}\label{training-our-rnn-model-for-text-generation}

With our RNN setup we can now train it! Lets begin by trying it out on a
small subset of the larger version. In the next cell we take the first
10,000 input/output pairs from our training database to learn on.

    \begin{Verbatim}[commandchars=\\\{\}]
{\color{incolor}In [{\color{incolor}28}]:} \PY{c+c1}{\PYZsh{} a small subset of our input/output pairs}
         \PY{n}{Xsmall} \PY{o}{=} \PY{n}{X}\PY{p}{[}\PY{p}{:}\PY{l+m+mi}{10000}\PY{p}{,}\PY{p}{:}\PY{p}{,}\PY{p}{:}\PY{p}{]}
         \PY{n}{ysmall} \PY{o}{=} \PY{n}{y}\PY{p}{[}\PY{p}{:}\PY{l+m+mi}{10000}\PY{p}{,}\PY{p}{:}\PY{p}{]}
\end{Verbatim}


    Now lets fit our model!

    \begin{Verbatim}[commandchars=\\\{\}]
{\color{incolor}In [{\color{incolor}29}]:} \PY{c+c1}{\PYZsh{} train the model}
         \PY{n}{model}\PY{o}{.}\PY{n}{fit}\PY{p}{(}\PY{n}{Xsmall}\PY{p}{,} \PY{n}{ysmall}\PY{p}{,} \PY{n}{batch\PYZus{}size}\PY{o}{=}\PY{l+m+mi}{500}\PY{p}{,} \PY{n}{epochs}\PY{o}{=}\PY{l+m+mi}{40}\PY{p}{,}\PY{n}{verbose} \PY{o}{=} \PY{l+m+mi}{1}\PY{p}{)}
         
         \PY{c+c1}{\PYZsh{} save weights}
         \PY{n}{model}\PY{o}{.}\PY{n}{save\PYZus{}weights}\PY{p}{(}\PY{l+s+s1}{\PYZsq{}}\PY{l+s+s1}{model\PYZus{}weights/best\PYZus{}RNN\PYZus{}small\PYZus{}textdata\PYZus{}weights.hdf5}\PY{l+s+s1}{\PYZsq{}}\PY{p}{)}
\end{Verbatim}


    \begin{Verbatim}[commandchars=\\\{\}]
Epoch 1/40
10000/10000 [==============================] - 41s 4ms/step - loss: 3.3227
Epoch 2/40
10000/10000 [==============================] - 40s 4ms/step - loss: 2.8422
Epoch 3/40
10000/10000 [==============================] - 40s 4ms/step - loss: 2.6591
Epoch 4/40
10000/10000 [==============================] - 43s 4ms/step - loss: 2.4555
Epoch 5/40
10000/10000 [==============================] - 42s 4ms/step - loss: 2.3347
Epoch 6/40
10000/10000 [==============================] - 41s 4ms/step - loss: 2.2218
Epoch 7/40
10000/10000 [==============================] - 42s 4ms/step - loss: 2.1222
Epoch 8/40
10000/10000 [==============================] - 43s 4ms/step - loss: 1.9904
Epoch 9/40
10000/10000 [==============================] - 43s 4ms/step - loss: 1.8918
Epoch 10/40
10000/10000 [==============================] - 42s 4ms/step - loss: 1.7788
Epoch 11/40
10000/10000 [==============================] - 43s 4ms/step - loss: 1.6682
Epoch 12/40
10000/10000 [==============================] - 42s 4ms/step - loss: 1.5541
Epoch 13/40
10000/10000 [==============================] - 42s 4ms/step - loss: 1.4062
Epoch 14/40
10000/10000 [==============================] - 42s 4ms/step - loss: 1.2580
Epoch 15/40
10000/10000 [==============================] - 43s 4ms/step - loss: 1.1305
Epoch 16/40
10000/10000 [==============================] - 44s 4ms/step - loss: 0.9880
Epoch 17/40
10000/10000 [==============================] - 42s 4ms/step - loss: 0.8538
Epoch 18/40
10000/10000 [==============================] - 42s 4ms/step - loss: 0.7226
Epoch 19/40
10000/10000 [==============================] - 42s 4ms/step - loss: 0.6261
Epoch 20/40
10000/10000 [==============================] - 42s 4ms/step - loss: 0.5236
Epoch 21/40
10000/10000 [==============================] - 42s 4ms/step - loss: 0.4601
Epoch 22/40
10000/10000 [==============================] - 42s 4ms/step - loss: 0.3932
Epoch 23/40
10000/10000 [==============================] - 42s 4ms/step - loss: 0.3324
Epoch 24/40
10000/10000 [==============================] - 42s 4ms/step - loss: 0.3113
Epoch 25/40
10000/10000 [==============================] - 42s 4ms/step - loss: 0.2709
Epoch 26/40
10000/10000 [==============================] - 42s 4ms/step - loss: 0.2510
Epoch 27/40
10000/10000 [==============================] - 42s 4ms/step - loss: 0.2212
Epoch 28/40
10000/10000 [==============================] - 42s 4ms/step - loss: 0.2043
Epoch 29/40
10000/10000 [==============================] - 42s 4ms/step - loss: 0.2027
Epoch 30/40
10000/10000 [==============================] - 42s 4ms/step - loss: 0.1816
Epoch 31/40
10000/10000 [==============================] - 42s 4ms/step - loss: 0.1680
Epoch 32/40
10000/10000 [==============================] - 42s 4ms/step - loss: 0.1713
Epoch 33/40
10000/10000 [==============================] - 42s 4ms/step - loss: 0.1534
Epoch 34/40
10000/10000 [==============================] - 42s 4ms/step - loss: 0.1691
Epoch 35/40
10000/10000 [==============================] - 42s 4ms/step - loss: 0.1389
Epoch 36/40
10000/10000 [==============================] - 42s 4ms/step - loss: 0.1427
Epoch 37/40
10000/10000 [==============================] - 42s 4ms/step - loss: 0.1480
Epoch 38/40
10000/10000 [==============================] - 42s 4ms/step - loss: 0.1387
Epoch 39/40
10000/10000 [==============================] - 42s 4ms/step - loss: 0.1384
Epoch 40/40
10000/10000 [==============================] - 42s 4ms/step - loss: 0.1372

    \end{Verbatim}

    How do we make a given number of predictions (characters) based on this
fitted model?

First we predict the next character after following any chunk of
characters in the text of length equal to our chosen window size. Then
we remove the first character in our input sequence and tack our
prediction onto the end. This gives us a slightly changed sequence of
inputs that still has length equal to the size of our window. We then
feed in this updated input sequence into the model to predict the
another character. Together then we have two predicted characters
following our original input sequence. Repeating this process N times
gives us N predicted characters.

In the next Python cell we provide you with a completed function that
does just this - it makes predictions when given a) a trained RNN model,
b) a subset of (window\_size) characters from the text, and c) a number
of characters to predict (to follow our input subset).

    \begin{Verbatim}[commandchars=\\\{\}]
{\color{incolor}In [{\color{incolor}30}]:} \PY{c+c1}{\PYZsh{} function that uses trained model to predict a desired number of future characters}
         \PY{k}{def} \PY{n+nf}{predict\PYZus{}next\PYZus{}chars}\PY{p}{(}\PY{n}{model}\PY{p}{,}\PY{n}{input\PYZus{}chars}\PY{p}{,}\PY{n}{num\PYZus{}to\PYZus{}predict}\PY{p}{)}\PY{p}{:}     
             \PY{c+c1}{\PYZsh{} create output}
             \PY{n}{predicted\PYZus{}chars} \PY{o}{=} \PY{l+s+s1}{\PYZsq{}}\PY{l+s+s1}{\PYZsq{}}
             \PY{k}{for} \PY{n}{i} \PY{o+ow}{in} \PY{n+nb}{range}\PY{p}{(}\PY{n}{num\PYZus{}to\PYZus{}predict}\PY{p}{)}\PY{p}{:}
                 \PY{c+c1}{\PYZsh{} convert this round\PYZsq{}s predicted characters to numerical input    }
                 \PY{n}{x\PYZus{}test} \PY{o}{=} \PY{n}{np}\PY{o}{.}\PY{n}{zeros}\PY{p}{(}\PY{p}{(}\PY{l+m+mi}{1}\PY{p}{,} \PY{n}{window\PYZus{}size}\PY{p}{,} \PY{n+nb}{len}\PY{p}{(}\PY{n}{chars}\PY{p}{)}\PY{p}{)}\PY{p}{)}
                 \PY{k}{for} \PY{n}{t}\PY{p}{,} \PY{n}{char} \PY{o+ow}{in} \PY{n+nb}{enumerate}\PY{p}{(}\PY{n}{input\PYZus{}chars}\PY{p}{)}\PY{p}{:}
                     \PY{n}{x\PYZus{}test}\PY{p}{[}\PY{l+m+mi}{0}\PY{p}{,} \PY{n}{t}\PY{p}{,} \PY{n}{chars\PYZus{}to\PYZus{}indices}\PY{p}{[}\PY{n}{char}\PY{p}{]}\PY{p}{]} \PY{o}{=} \PY{l+m+mf}{1.}
         
                 \PY{c+c1}{\PYZsh{} make this round\PYZsq{}s prediction}
                 \PY{n}{test\PYZus{}predict} \PY{o}{=} \PY{n}{model}\PY{o}{.}\PY{n}{predict}\PY{p}{(}\PY{n}{x\PYZus{}test}\PY{p}{,}\PY{n}{verbose} \PY{o}{=} \PY{l+m+mi}{0}\PY{p}{)}\PY{p}{[}\PY{l+m+mi}{0}\PY{p}{]}
         
                 \PY{c+c1}{\PYZsh{} translate numerical prediction back to characters}
                 \PY{n}{r} \PY{o}{=} \PY{n}{np}\PY{o}{.}\PY{n}{argmax}\PY{p}{(}\PY{n}{test\PYZus{}predict}\PY{p}{)}                           \PY{c+c1}{\PYZsh{} predict class of each test input}
                 \PY{n}{d} \PY{o}{=} \PY{n}{indices\PYZus{}to\PYZus{}chars}\PY{p}{[}\PY{n}{r}\PY{p}{]} 
         
                 \PY{c+c1}{\PYZsh{} update predicted\PYZus{}chars and input}
                 \PY{n}{predicted\PYZus{}chars}\PY{o}{+}\PY{o}{=}\PY{n}{d}
                 \PY{n}{input\PYZus{}chars}\PY{o}{+}\PY{o}{=}\PY{n}{d}
                 \PY{n}{input\PYZus{}chars} \PY{o}{=} \PY{n}{input\PYZus{}chars}\PY{p}{[}\PY{l+m+mi}{1}\PY{p}{:}\PY{p}{]}
             \PY{k}{return} \PY{n}{predicted\PYZus{}chars}
\end{Verbatim}


    With your trained model try a few subsets of the complete text as input
- note the length of each must be exactly equal to the window size. For
each subset use the function above to predict the next 100 characters
that follow each input.

    \begin{Verbatim}[commandchars=\\\{\}]
{\color{incolor}In [{\color{incolor}31}]:} \PY{c+c1}{\PYZsh{} TODO: choose an input sequence and use the prediction function in the previous Python cell to predict 100 characters following it}
         \PY{c+c1}{\PYZsh{} get an appropriately sized chunk of characters from the text}
         \PY{n}{start\PYZus{}inds} \PY{o}{=} \PY{p}{[}\PY{l+m+mi}{10}\PY{p}{,} \PY{l+m+mi}{24}\PY{p}{,} \PY{l+m+mi}{45}\PY{p}{,} \PY{l+m+mi}{103}\PY{p}{,} \PY{l+m+mi}{143}\PY{p}{]}
         
         \PY{c+c1}{\PYZsh{} load in weights}
         \PY{n}{model}\PY{o}{.}\PY{n}{load\PYZus{}weights}\PY{p}{(}\PY{l+s+s1}{\PYZsq{}}\PY{l+s+s1}{model\PYZus{}weights/best\PYZus{}RNN\PYZus{}small\PYZus{}textdata\PYZus{}weights.hdf5}\PY{l+s+s1}{\PYZsq{}}\PY{p}{)}
         \PY{k}{for} \PY{n}{s} \PY{o+ow}{in} \PY{n}{start\PYZus{}inds}\PY{p}{:}
             \PY{n}{start\PYZus{}index} \PY{o}{=} \PY{n}{s}
             \PY{n}{input\PYZus{}chars} \PY{o}{=} \PY{n}{text}\PY{p}{[}\PY{n}{start\PYZus{}index}\PY{p}{:} \PY{n}{start\PYZus{}index} \PY{o}{+} \PY{n}{window\PYZus{}size}\PY{p}{]}
         
             \PY{c+c1}{\PYZsh{} use the prediction function}
             \PY{n}{predict\PYZus{}input} \PY{o}{=} \PY{n}{predict\PYZus{}next\PYZus{}chars}\PY{p}{(}\PY{n}{model}\PY{p}{,}\PY{n}{input\PYZus{}chars}\PY{p}{,}\PY{n}{num\PYZus{}to\PYZus{}predict} \PY{o}{=} \PY{l+m+mi}{100}\PY{p}{)}
         
             \PY{c+c1}{\PYZsh{} print out input characters}
             \PY{n+nb}{print}\PY{p}{(}\PY{l+s+s1}{\PYZsq{}}\PY{l+s+s1}{\PYZhy{}\PYZhy{}\PYZhy{}\PYZhy{}\PYZhy{}\PYZhy{}\PYZhy{}\PYZhy{}\PYZhy{}\PYZhy{}\PYZhy{}\PYZhy{}\PYZhy{}\PYZhy{}\PYZhy{}\PYZhy{}\PYZhy{}\PYZhy{}}\PY{l+s+s1}{\PYZsq{}}\PY{p}{)}
             \PY{n}{input\PYZus{}line} \PY{o}{=} \PY{l+s+s1}{\PYZsq{}}\PY{l+s+s1}{input chars = }\PY{l+s+s1}{\PYZsq{}} \PY{o}{+} \PY{l+s+s1}{\PYZsq{}}\PY{l+s+se}{\PYZbs{}n}\PY{l+s+s1}{\PYZsq{}} \PY{o}{+}  \PY{n}{input\PYZus{}chars} \PY{o}{+} \PY{l+s+s1}{\PYZsq{}}\PY{l+s+s1}{\PYZdq{}}\PY{l+s+s1}{\PYZsq{}} \PY{o}{+} \PY{l+s+s1}{\PYZsq{}}\PY{l+s+se}{\PYZbs{}n}\PY{l+s+s1}{\PYZsq{}}
             \PY{n+nb}{print}\PY{p}{(}\PY{n}{input\PYZus{}line}\PY{p}{)}
         
             \PY{c+c1}{\PYZsh{} print out predicted characters}
             \PY{n}{line} \PY{o}{=} \PY{l+s+s1}{\PYZsq{}}\PY{l+s+s1}{predicted chars = }\PY{l+s+s1}{\PYZsq{}} \PY{o}{+} \PY{l+s+s1}{\PYZsq{}}\PY{l+s+se}{\PYZbs{}n}\PY{l+s+s1}{\PYZsq{}} \PY{o}{+}  \PY{n}{predict\PYZus{}input} \PY{o}{+} \PY{l+s+s1}{\PYZsq{}}\PY{l+s+s1}{\PYZdq{}}\PY{l+s+s1}{\PYZsq{}} \PY{o}{+} \PY{l+s+s1}{\PYZsq{}}\PY{l+s+se}{\PYZbs{}n}\PY{l+s+s1}{\PYZsq{}}
             \PY{n+nb}{print}\PY{p}{(}\PY{n}{line}\PY{p}{)}
\end{Verbatim}


    \begin{Verbatim}[commandchars=\\\{\}]
------------------
input chars = 
she eclipses and predominates the whole of her sex. it was not that he felt any emotion akin to love"

predicted chars = 
 and in a craiked but in the lomes and confrulid, stronging he crieed for the cabidrape.  then i, as"

------------------
input chars = 
nd predominates the whole of her sex. it was not that he felt any emotion akin to love for irene adl"

predicted chars = 
ers with a belf drom hoh sook aid contspliap.  but you was the temped of you so be grould her forded"

------------------
input chars = 
hole of her sex. it was not that he felt any emotion akin to love for irene adler. all emotions, and"

predicted chars = 
 i have all thr i have all i will she sope and to atyour four inth mind beer what gerable to his fit"

------------------
input chars = 
to love for irene adler. all emotions, and that one particularly, were abhorrent to his cold, precis"

predicted chars = 
tly in the prosestcangen an upon a couct a looghing with a cryelif to the strieds of a craiget for t"

------------------
input chars = 
nd that one particularly, were abhorrent to his cold, precise but admirably balanced mind. he was, i"

predicted chars = 
 an the coner mistry, and, poren street fire of to dorn nor of his long and murh, bohe and the adler"


    \end{Verbatim}

    This looks ok, but not great. Now lets try the same experiment with a
larger chunk of the data - with the first 100,000 input/output pairs.

Tuning RNNs for a typical character dataset like the one we will use
here is a computationally intensive endeavour and thus timely on a
typical CPU. Using a reasonably sized cloud-based GPU can speed up
training by a factor of 10. Also because of the long training time it is
highly recommended that you carefully write the output of each step of
your process to file. This is so that all of your results are saved even
if you close the web browser you're working out of, as the processes
will continue processing in the background but variables/output in the
notebook system will not update when you open it again.

In the next cell we show you how to create a text file in Python and
record data to it. This sort of setup can be used to record your final
predictions.

    \begin{Verbatim}[commandchars=\\\{\}]
{\color{incolor}In [{\color{incolor}32}]:} \PY{c+c1}{\PYZsh{}\PYZsh{}\PYZsh{} A simple way to write output to file}
         \PY{n}{f} \PY{o}{=} \PY{n+nb}{open}\PY{p}{(}\PY{l+s+s1}{\PYZsq{}}\PY{l+s+s1}{my\PYZus{}test\PYZus{}output.txt}\PY{l+s+s1}{\PYZsq{}}\PY{p}{,} \PY{l+s+s1}{\PYZsq{}}\PY{l+s+s1}{w}\PY{l+s+s1}{\PYZsq{}}\PY{p}{)}              \PY{c+c1}{\PYZsh{} create an output file to write too}
         \PY{n}{f}\PY{o}{.}\PY{n}{write}\PY{p}{(}\PY{l+s+s1}{\PYZsq{}}\PY{l+s+s1}{this is only a test }\PY{l+s+s1}{\PYZsq{}} \PY{o}{+} \PY{l+s+s1}{\PYZsq{}}\PY{l+s+se}{\PYZbs{}n}\PY{l+s+s1}{\PYZsq{}}\PY{p}{)}           \PY{c+c1}{\PYZsh{} print some output text}
         \PY{n}{x} \PY{o}{=} \PY{l+m+mi}{2}
         \PY{n}{f}\PY{o}{.}\PY{n}{write}\PY{p}{(}\PY{l+s+s1}{\PYZsq{}}\PY{l+s+s1}{the value of x is }\PY{l+s+s1}{\PYZsq{}} \PY{o}{+} \PY{n+nb}{str}\PY{p}{(}\PY{n}{x}\PY{p}{)} \PY{o}{+} \PY{l+s+s1}{\PYZsq{}}\PY{l+s+se}{\PYZbs{}n}\PY{l+s+s1}{\PYZsq{}}\PY{p}{)}    \PY{c+c1}{\PYZsh{} record a variable value}
         \PY{n}{f}\PY{o}{.}\PY{n}{close}\PY{p}{(}\PY{p}{)}     
         
         \PY{c+c1}{\PYZsh{} print out the contents of my\PYZus{}test\PYZus{}output.txt}
         \PY{n}{f} \PY{o}{=} \PY{n+nb}{open}\PY{p}{(}\PY{l+s+s1}{\PYZsq{}}\PY{l+s+s1}{my\PYZus{}test\PYZus{}output.txt}\PY{l+s+s1}{\PYZsq{}}\PY{p}{,} \PY{l+s+s1}{\PYZsq{}}\PY{l+s+s1}{r}\PY{l+s+s1}{\PYZsq{}}\PY{p}{)}              \PY{c+c1}{\PYZsh{} create an output file to write too}
         \PY{n}{f}\PY{o}{.}\PY{n}{read}\PY{p}{(}\PY{p}{)}
\end{Verbatim}


\begin{Verbatim}[commandchars=\\\{\}]
{\color{outcolor}Out[{\color{outcolor}32}]:} 'this is only a test \textbackslash{}nthe value of x is 2\textbackslash{}n'
\end{Verbatim}
            
    With this recording devices we can now more safely perform experiments
on larger portions of the text. In the next cell we will use the first
100,000 input/output pairs to train our RNN model.

    First we fit our model to the dataset, then generate text using the
trained model in precisely the same generation method applied before on
the small dataset.

\textbf{Note:} your generated words should be - by and large - more
realistic than with the small dataset, but you won't be able to generate
perfect English sentences even with this amount of data. A rule of
thumb: your model is working well if you generate sentences that largely
contain real English words.

    \begin{Verbatim}[commandchars=\\\{\}]
{\color{incolor}In [{\color{incolor}33}]:} \PY{c+c1}{\PYZsh{}\PYZsh{}\PYZsh{}\PYZsh{}\PYZsh{} a small subset of our input/output pairs}
         \PY{n}{Xlarge} \PY{o}{=} \PY{n}{X}\PY{p}{[}\PY{p}{:}\PY{l+m+mi}{100000}\PY{p}{,}\PY{p}{:}\PY{p}{,}\PY{p}{:}\PY{p}{]}
         \PY{n}{ylarge} \PY{o}{=} \PY{n}{y}\PY{p}{[}\PY{p}{:}\PY{l+m+mi}{100000}\PY{p}{,}\PY{p}{:}\PY{p}{]}
         
         \PY{c+c1}{\PYZsh{} TODO: fit to our larger dataset}
         \PY{n}{model}\PY{o}{.}\PY{n}{fit}\PY{p}{(}\PY{n}{Xlarge}\PY{p}{,} \PY{n}{ylarge}\PY{p}{,} \PY{n}{batch\PYZus{}size}\PY{o}{=}\PY{l+m+mi}{500}\PY{p}{,} \PY{n}{epochs}\PY{o}{=}\PY{l+m+mi}{50}\PY{p}{,} \PY{n}{verbose}\PY{o}{=}\PY{l+m+mi}{1}\PY{p}{)}
         
         \PY{c+c1}{\PYZsh{} save weights}
         \PY{n}{model}\PY{o}{.}\PY{n}{save\PYZus{}weights}\PY{p}{(}\PY{l+s+s1}{\PYZsq{}}\PY{l+s+s1}{model\PYZus{}weights/best\PYZus{}RNN\PYZus{}large\PYZus{}textdata\PYZus{}weights.hdf5}\PY{l+s+s1}{\PYZsq{}}\PY{p}{)}
\end{Verbatim}


    \begin{Verbatim}[commandchars=\\\{\}]
Epoch 1/50
100000/100000 [==============================] - 418s 4ms/step - loss: 1.8534
Epoch 2/50
100000/100000 [==============================] - 418s 4ms/step - loss: 1.5352
Epoch 3/50
100000/100000 [==============================] - 418s 4ms/step - loss: 1.4074
Epoch 4/50
100000/100000 [==============================] - 418s 4ms/step - loss: 1.3155
Epoch 5/50
100000/100000 [==============================] - 418s 4ms/step - loss: 1.2395
Epoch 6/50
100000/100000 [==============================] - 418s 4ms/step - loss: 1.1772
Epoch 7/50
100000/100000 [==============================] - 417s 4ms/step - loss: 1.1267
Epoch 8/50
100000/100000 [==============================] - 417s 4ms/step - loss: 1.0857
Epoch 9/50
100000/100000 [==============================] - 417s 4ms/step - loss: 1.0503
Epoch 10/50
100000/100000 [==============================] - 417s 4ms/step - loss: 1.0196
Epoch 11/50
100000/100000 [==============================] - 417s 4ms/step - loss: 0.9946
Epoch 12/50
100000/100000 [==============================] - 417s 4ms/step - loss: 0.9745
Epoch 13/50
100000/100000 [==============================] - 417s 4ms/step - loss: 0.9570
Epoch 14/50
100000/100000 [==============================] - 419s 4ms/step - loss: 0.9384
Epoch 15/50
100000/100000 [==============================] - 418s 4ms/step - loss: 0.9214
Epoch 16/50
100000/100000 [==============================] - 418s 4ms/step - loss: 0.9057
Epoch 17/50
100000/100000 [==============================] - 418s 4ms/step - loss: 0.8902
Epoch 18/50
100000/100000 [==============================] - 418s 4ms/step - loss: 0.8776
Epoch 19/50
100000/100000 [==============================] - 417s 4ms/step - loss: 0.8661
Epoch 20/50
100000/100000 [==============================] - 418s 4ms/step - loss: 0.8550
Epoch 21/50
100000/100000 [==============================] - 418s 4ms/step - loss: 0.8425
Epoch 22/50
100000/100000 [==============================] - 418s 4ms/step - loss: 0.8276
Epoch 23/50
100000/100000 [==============================] - 418s 4ms/step - loss: 0.8205
Epoch 24/50
100000/100000 [==============================] - 419s 4ms/step - loss: 0.8107
Epoch 25/50
100000/100000 [==============================] - 419s 4ms/step - loss: 0.7991
Epoch 26/50
100000/100000 [==============================] - 420s 4ms/step - loss: 0.7886
Epoch 27/50
100000/100000 [==============================] - 419s 4ms/step - loss: 0.7790
Epoch 28/50
100000/100000 [==============================] - 420s 4ms/step - loss: 0.7668
Epoch 29/50
100000/100000 [==============================] - 419s 4ms/step - loss: 0.7595
Epoch 30/50
100000/100000 [==============================] - 419s 4ms/step - loss: 0.7528
Epoch 31/50
100000/100000 [==============================] - 418s 4ms/step - loss: 0.7432
Epoch 32/50
100000/100000 [==============================] - 419s 4ms/step - loss: 0.7326
Epoch 33/50
100000/100000 [==============================] - 421s 4ms/step - loss: 0.7260
Epoch 34/50
100000/100000 [==============================] - 419s 4ms/step - loss: 0.7189
Epoch 35/50
100000/100000 [==============================] - 418s 4ms/step - loss: 0.7071
Epoch 36/50
100000/100000 [==============================] - 419s 4ms/step - loss: 0.7004
Epoch 37/50
100000/100000 [==============================] - 418s 4ms/step - loss: 0.6939
Epoch 38/50
100000/100000 [==============================] - 418s 4ms/step - loss: 0.6854
Epoch 39/50
100000/100000 [==============================] - 418s 4ms/step - loss: 0.6761
Epoch 40/50
100000/100000 [==============================] - 408s 4ms/step - loss: 0.6689
Epoch 41/50
100000/100000 [==============================] - 408s 4ms/step - loss: 0.6604
Epoch 42/50
100000/100000 [==============================] - 407s 4ms/step - loss: 0.6541
Epoch 43/50
100000/100000 [==============================] - 408s 4ms/step - loss: 0.6509
Epoch 44/50
100000/100000 [==============================] - 408s 4ms/step - loss: 0.6438
Epoch 45/50
100000/100000 [==============================] - 408s 4ms/step - loss: 0.6368
Epoch 46/50
100000/100000 [==============================] - 408s 4ms/step - loss: 0.6289
Epoch 47/50
100000/100000 [==============================] - 408s 4ms/step - loss: 0.6254
Epoch 48/50
100000/100000 [==============================] - 410s 4ms/step - loss: 0.6181
Epoch 49/50
100000/100000 [==============================] - 408s 4ms/step - loss: 0.6165
Epoch 50/50
100000/100000 [==============================] - 408s 4ms/step - loss: 0.6055

    \end{Verbatim}

    \begin{Verbatim}[commandchars=\\\{\}]
{\color{incolor}In [{\color{incolor}34}]:} \PY{c+c1}{\PYZsh{} TODO: choose an input sequence and use the prediction function in the previous Python cell to predict 100 characters following it}
         \PY{c+c1}{\PYZsh{} get an appropriately sized chunk of characters from the text}
         \PY{n}{start\PYZus{}inds} \PY{o}{=} \PY{p}{[}\PY{l+m+mi}{100}\PY{p}{,} \PY{l+m+mi}{224}\PY{p}{,} \PY{l+m+mi}{350}\PY{p}{,} \PY{l+m+mi}{403}\PY{p}{,} \PY{l+m+mi}{1143}\PY{p}{]}
         
         \PY{c+c1}{\PYZsh{} save output}
         \PY{n}{f} \PY{o}{=} \PY{n+nb}{open}\PY{p}{(}\PY{l+s+s1}{\PYZsq{}}\PY{l+s+s1}{text\PYZus{}gen\PYZus{}output/RNN\PYZus{}large\PYZus{}textdata\PYZus{}output.txt}\PY{l+s+s1}{\PYZsq{}}\PY{p}{,} \PY{l+s+s1}{\PYZsq{}}\PY{l+s+s1}{w}\PY{l+s+s1}{\PYZsq{}}\PY{p}{)}  \PY{c+c1}{\PYZsh{} create an output file to write too}
         
         \PY{c+c1}{\PYZsh{} load weights}
         \PY{n}{model}\PY{o}{.}\PY{n}{load\PYZus{}weights}\PY{p}{(}\PY{l+s+s1}{\PYZsq{}}\PY{l+s+s1}{model\PYZus{}weights/best\PYZus{}RNN\PYZus{}large\PYZus{}textdata\PYZus{}weights.hdf5}\PY{l+s+s1}{\PYZsq{}}\PY{p}{)}
         \PY{k}{for} \PY{n}{s} \PY{o+ow}{in} \PY{n}{start\PYZus{}inds}\PY{p}{:}
             \PY{n}{start\PYZus{}index} \PY{o}{=} \PY{n}{s}
             \PY{n}{input\PYZus{}chars} \PY{o}{=} \PY{n}{text}\PY{p}{[}\PY{n}{start\PYZus{}index}\PY{p}{:} \PY{n}{start\PYZus{}index} \PY{o}{+} \PY{n}{window\PYZus{}size}\PY{p}{]}
         
             \PY{c+c1}{\PYZsh{} use the prediction function}
             \PY{n}{predict\PYZus{}input} \PY{o}{=} \PY{n}{predict\PYZus{}next\PYZus{}chars}\PY{p}{(}\PY{n}{model}\PY{p}{,}\PY{n}{input\PYZus{}chars}\PY{p}{,}\PY{n}{num\PYZus{}to\PYZus{}predict} \PY{o}{=} \PY{l+m+mi}{100}\PY{p}{)}
         
             \PY{c+c1}{\PYZsh{} print out input characters}
             \PY{n}{line} \PY{o}{=} \PY{l+s+s1}{\PYZsq{}}\PY{l+s+s1}{\PYZhy{}\PYZhy{}\PYZhy{}\PYZhy{}\PYZhy{}\PYZhy{}\PYZhy{}\PYZhy{}\PYZhy{}\PYZhy{}\PYZhy{}\PYZhy{}\PYZhy{}\PYZhy{}\PYZhy{}\PYZhy{}\PYZhy{}\PYZhy{}\PYZhy{}}\PY{l+s+s1}{\PYZsq{}} \PY{o}{+} \PY{l+s+s1}{\PYZsq{}}\PY{l+s+se}{\PYZbs{}n}\PY{l+s+s1}{\PYZsq{}}
             \PY{n+nb}{print}\PY{p}{(}\PY{n}{line}\PY{p}{)}
             \PY{n}{f}\PY{o}{.}\PY{n}{write}\PY{p}{(}\PY{n}{line}\PY{p}{)}
         
             \PY{n}{input\PYZus{}line} \PY{o}{=} \PY{l+s+s1}{\PYZsq{}}\PY{l+s+s1}{input chars = }\PY{l+s+s1}{\PYZsq{}} \PY{o}{+} \PY{l+s+s1}{\PYZsq{}}\PY{l+s+se}{\PYZbs{}n}\PY{l+s+s1}{\PYZsq{}} \PY{o}{+}  \PY{n}{input\PYZus{}chars} \PY{o}{+} \PY{l+s+s1}{\PYZsq{}}\PY{l+s+s1}{\PYZdq{}}\PY{l+s+s1}{\PYZsq{}} \PY{o}{+} \PY{l+s+s1}{\PYZsq{}}\PY{l+s+se}{\PYZbs{}n}\PY{l+s+s1}{\PYZsq{}}
             \PY{n+nb}{print}\PY{p}{(}\PY{n}{input\PYZus{}line}\PY{p}{)}
             \PY{n}{f}\PY{o}{.}\PY{n}{write}\PY{p}{(}\PY{n}{input\PYZus{}line}\PY{p}{)}
         
             \PY{c+c1}{\PYZsh{} print out predicted characters}
             \PY{n}{predict\PYZus{}line} \PY{o}{=} \PY{l+s+s1}{\PYZsq{}}\PY{l+s+s1}{predicted chars = }\PY{l+s+s1}{\PYZsq{}} \PY{o}{+} \PY{l+s+s1}{\PYZsq{}}\PY{l+s+se}{\PYZbs{}n}\PY{l+s+s1}{\PYZsq{}} \PY{o}{+}  \PY{n}{predict\PYZus{}input} \PY{o}{+} \PY{l+s+s1}{\PYZsq{}}\PY{l+s+s1}{\PYZdq{}}\PY{l+s+s1}{\PYZsq{}} \PY{o}{+} \PY{l+s+s1}{\PYZsq{}}\PY{l+s+se}{\PYZbs{}n}\PY{l+s+s1}{\PYZsq{}}
             \PY{n+nb}{print}\PY{p}{(}\PY{n}{predict\PYZus{}line}\PY{p}{)}
             \PY{n}{f}\PY{o}{.}\PY{n}{write}\PY{p}{(}\PY{n}{predict\PYZus{}line}\PY{p}{)}
         \PY{n}{f}\PY{o}{.}\PY{n}{close}\PY{p}{(}\PY{p}{)}
\end{Verbatim}


    \begin{Verbatim}[commandchars=\\\{\}]
-------------------

input chars = 
in to love for irene adler. all emotions, and that one particularly, were abhorrent to his cold, pre"

predicted chars = 
cious to crip into the room which was her dark that the word by gatut all the passabbliced the stair"

-------------------

input chars = 
ced mind. he was, i take it, the most perfect reasoning and observing machine that the world has see"

predicted chars = 
 when of the sight, that the police crip, and what do you tell me companion at the other and said he"

-------------------

input chars = 
 have placed himself in a false position. he never spoke of the softer passions, save with a gibe an"

predicted chars = 
d the certainly said than of last simple in his confider that we perhaps where the point and the fir"

-------------------

input chars = 
oke of the softer passions, save with a gibe and a sneer. they were admirable things for the observe"

predicted chars = 
d my sister silently for hold in the case for the second flock. it silent and to find the stairs, th"

-------------------

input chars = 
happiness, and the home centred interests which rise up around the man who first finds himself maste"

predicted chars = 
d upon the sidested.  the goose upor before herely seemed to have the other side of the fire that yo"


    \end{Verbatim}


    % Add a bibliography block to the postdoc
    
    
    
    \end{document}
